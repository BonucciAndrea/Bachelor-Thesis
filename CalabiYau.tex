\documentclass{standalone}
\usepackage{asymptote}

\begin{document}

\begin{asy}[width=6.5cm,height=6.5cm,]
import graph3;

size3(200);
currentprojection=orthographic(3,3,2);
currentlight=light(8,10,20);

int k1, k2, n = 5;
real alpha = 0.3*pi;

// cross section of the quintic 6D Calabi-Yau manifold
triple cy(pair z) {  

  pair z1, z2;

  if(z==(0,0)) {
    z1 = exp(2*pi*I*k1/n);
    z2 = 0;
  } else {
    z1 = exp(2*pi*I*k1/n)*exp(log(cos(I*z))*2/n);
    z2 = exp(2*pi*I*k2/n)*exp(log(-I*sin(I*z))*2/n);
  }

  return (z2.x, cos(alpha)*z1.y + sin(alpha)*z2.y, z1.x);

}

for(k1=0; k1<n; ++k1) {
  for(k2=0; k2<n; ++k2) {
    surface s = surface(cy,(-1,0),(1,0.5*pi),20,20);
    draw(s,mediumgray + black);
  }
}

\end{asy}

\end{document}

enter image description here