\documentclass[12pt,vu]{adammath}
\usepackage[utf8]{inputenc}
\usepackage{amsmath}
\usepackage{amssymb}
\usepackage{asymptote}
\usepackage{amsthm}
\usepackage{pgfplots}
\usepackage{mathtools}
\usepackage{multirow}
\usepackage{float}
\usepackage{tikz}
\usetikzlibrary{backgrounds}
\usepackage{quiver}
\usepackage{verbatim}
\usepackage{standalone}
\usepackage{lipsum}
\usepackage{unicode-math}
\usepackage{yhmath}
\usepackage{graphicx}

\newcommand{\KP}[1]{%
  \begin{tikzpicture}[baseline=-\dimexpr\fontdimen22\textfont2\relax]
  #1
  \end{tikzpicture}%
}
\newcommand{\KPA}{%
  \KP{\filldraw[color=gray, fill=none, thick] circle (0.3);}%
}
\newcommand{\KPB}{%
  \KP{
    \draw[color=gray,thick] (-0.3,0.3) -- (0.3,-0.3);
    \draw[color=gray,thick] (-0.3,-0.3) -- (-0.05,-0.05);
    \draw[color=gray,thick] (0.05,0.05) -- (0.3,0.3);
  }%
}
\newcommand{\KPC}{%
  \KP{%
    \draw[color=gray,thick] (-0.3,0.3) .. controls (0,-0.05) .. (0.3,0.3);
    \draw[color=gray,thick] (-0.3,-0.3) .. controls (0,0.05) .. (0.3,-0.3);
  }%
}
\newcommand{\KPD}{%
  \KP{%
    \draw[color=gray,thick] (-0.3,-0.3) .. controls (0.05,0) .. (-0.3,0.3);
    \draw[color=gray,thick] (0.3,-0.3) .. controls (-0.05,0) .. (0.3,0.3);
  }%
}


\makeatletter
\newcommand*{\centerfloat}{%
  \parindent \z@
  \leftskip \z@ \@plus 1fil \@minus \textwidth
  \rightskip\leftskip
  \parfillskip \z@skip}
\makeatother



\usepackage{mathrsfs}
\usepackage{hyperref}
\usepackage[
backend=biber,
style=alphabetic,
]{biblatex}
\bibliography{Sources}
\nocite{*}

\pgfdeclarelayer{bg}    % declare background layer
\pgfsetlayers{bg,main}

\def\s{\ensuremath\mathcal{S}}
\newcommand{\pal}[1]{{#1}_{\leftrightarrow}}

\theoremstyle{theorem}
\newtheorem{theorem}{Theorem}[chapter] 

\theoremstyle{corollary}
\newtheorem{corollary}[theorem]{Corollary}

\theoremstyle{conjecture}
\newtheorem{conjecture}[theorem]{Conjecture}

\theoremstyle{proposition}
\newtheorem{proposition}[theorem]{Proposition}

\theoremstyle{definition}
\newtheorem{definition}[theorem]{Definition}
\newtheorem{example}[theorem]{Example}

\theoremstyle{remark}
\newtheorem{remark}[theorem]{Remark}

\title{Combinatorial aspects of surface Cluster algebras and applications to Frobenius' conjecture}
\abstracttitle{Combinatorial aspects of surface Cluster algebras and applications to Frobenius' conjecture}
\author[a.bonucci@student.vu.nl, 2694276]{Andrea Bonucci}
\supervisor{Dr. {\.{I}}lke {\c{C}}anak{\c{c}}{\i}}
\secondgrader{Dr. Senja Barthel}
\documentclass{standalone}
%\usepackage[utf8]{inputenc}
\usepackage[english]{babel}
\usepackage{amsmath}
\usepackage{amsfonts}
\usepackage{xcolor}
\ifluatex\usepackage{luacode}\fi
\usepackage{amssymb}
%\usepackage[utf8]{inputenc}
\usepackage{pgfplots}
\pgfplotsset{compat=1.15}
\usepackage{hyperref}

\usetikzlibrary{matrix,decorations.pathreplacing,calc,shapes,fit}
\usetikzlibrary{backgrounds}
\usepackage{mathrsfs}
\def\curlyG{\mathscr{G}}

%%%%%%%%%%%%%%%%%%%%%%%%%%%%%%%%%%%%%%%%%%%%%%
\newcommand{\colorzero}[1]{\color{gray}{#1}}
\newcommand{\colorone}[1]{\color{blue}{#1}}
%
\pgfkeys{tikz/mymatrixbrace/.style={decorate,thick}}
\ifluatex
\begin{luacode*}
	function signtile(x,y,c)
		if c==0 then 
			s="\\color{red}$-$" 
		else 
			s='\\color{blue}{$+$}' 
		end 
		tex.print("\\node[text width=1.5ex,fill=white,align=center] at ("..x..","..y.."){"..s.."};")
	end
	
	function signtiles(x,y,side)
		if side==0 then 
			c1= 0 
			c2= 1
		else 
			c1= 1 
			c2= 0
		end
		signtile(x,y-0.5,c1);
		signtile(x+0.5,y,c1);
		signtile(x,y+0.5,c2);
		signtile(x-0.5,y,c2);
	end	
	
-- as	 : "nenenenen"  e:east n:north
-- type:	1: G_n 
-- 		2: G_n e_n
-- 		3: G_n signs
-- 		4: signs 
-- 		5:
-- 		6:
-- 		7:
	function tikzsnake(as,type)
		type = type or 1
		x=0
		y=0
		sign=0
		asl=0
		asl=string.len(as)
		if type==1 or type==2 or type==3 then
			tex.print("\\node [square={snake square}] at (0,0) (B0) {$G_{1}$};")
		else
			tex.print("\\node [square={snake square}] at (0,0) (B0) {};")
		end
		if type==3 then
			signtiles(0,0,sign)
		end
		if type==4 then
			if asl>0 then
				if 'n' == string.sub(as,1,1) then
					signtile(-0.5,0,sign)
					sign=1
				else
					signtile(0,-0.5,sign)
				end
			else
				signtile(0,-0.5,sign)
				signtile(0.5,0,sign)
			end	
		end		
		for i=1,asl do
			c=string.sub(as,i,i)
			if c=='e' then x=x+1 end 
			if c=='n' then y=y+1 end
			if type==1 or type==2 or type== 3 then 
				tex.print("\\node [square={snake square}] at ("..x..","..y..") (B"..i..") {$G_{"..(i+1).."}$};")
			else
				tex.print("\\node [square={snake square}] at ("..x..","..y..") (B"..i..") {};")
			end
			if type==2 then
				if c=='e' then
					tex.print("\\node[text width=1.5ex,fill=white,align=center] at ("..(x-0.5)..","..y.."){$e_{"..i.."}$};")
				end
				if c=='n' then
					tex.print("\\node[text width=1.5ex,fill=white,align=center] at ("..x..","..(y-0.5).."){$e_{"..i.."}$};")
				end
			end
			if type==3 then
				if sign==1 then sign=0 else sign=1 end
				signtiles(x,y,sign)
			end
			if type==4 then
				if i==1 then
					if c=='n' then
						if sign==1 then sign=0 else sign=1 end
					end
				else
					if c==string.sub(as,i-1,i-1) then
						if sign==1 then sign=0 else sign=1 end
					end
				end
				if c=='e'  then
					signtile(x-0.5,y,sign)
				else
					signtile(x,y-0.5,sign)
				end
				if i==asl then 
					if c=='e'  then
						signtile(x,y+0.5,sign)
					else
						signtile(x+0.5,y,sign)
					end
				end
			end
		end
	end
\end{luacode*}
\fi
%
%
\tikzset{
    square/.style={%
        draw=none,
        circle,
        minimum size=1,
        append after command={%
            \pgfextra \draw[#1] ([shift={(-0.5,0.5)}]\tikzlastnode.center)
            --([shift={(0.5,0.5)}]\tikzlastnode.center)
            --([shift={(0.5,-0.5)}]\tikzlastnode.center)
            --([shift={(-0.5,-0.5)}]\tikzlastnode.center)
            --([shift={(-0.5,0.5)}]\tikzlastnode.center)
            ;\endpgfextra}
    },
    square/.default=black
}
\pgfkeys{tikz/snake square/.style={black,thick,fill=none}}

\begin{document}

\programme{Bachelor Mathematics}
\track{Bachelor Thesis}

\maketitle

\begin{abstract}
    Since Frobenius stated his conjecture on the uniqueness of Markov triples in 1913, many have attempted to crack it; and in doing so uncovered essential knowledge about the conjecture. In this report, we seek to explore various techniques within Cluster algebra and utilize them in order to better understand the behaviour of Markov numbers. We use \emph{palindromification} of continued fractions and connect them to the idea of \emph{snake graph} to attain a reformulation of Frobenius' conjecture in Cluster algebraic terms. Consequently, we apply \emph{Skein relations} within the natural number lattice $\mathbb{N}\times \mathbb{N}$ to define \emph{left} and \emph{right deformations} around lattice points to provide a few result on the ordering of Markov numbers; and prove a conjecture posed by Aigner in \cite{A}. 
\end{abstract}
\newpage 
\tableofcontents
\newpage
\chapter*{Introduction}
Cluster algebras were first introduced in 2002 by Sergey Fomin and Andrei Zelevinsky in \cite{FZ1}. These have been getting increasingly more attention and are found in many areas of mathematics; such as combinatorics, as we will see throughout this report, representation theory, tropical geometry, and many more. Additionally, there are several applications in other disclipines; for example in theoretical physics, we have \emph{Calabi-Yau manifolds}  (cover page) and a concept named \emph{Seiberg Duality} \cite{Bao}. 

In the first chapter, we will outline the formal definition of Cluster algebras in purely algebraic terms. We begin by defining the ambient field, a \emph{tropical semifield}; then we describe the three components of a Cluster algebra, namely, the \emph{initial cluster} $\mathbf{x}$, the \emph{initial coefficients} $\mathbf{y}$ and the cluster quiver $\mathcal{Q}$ (a type of directed graph). It is worth noting that Fomin and Zelevinsky first introduced cluster algebras via \emph{skew-symmetric} matrices; which then has an associated quiver. 

From the above, we can then define the process through which \emph{seeds} generate the corresponding cluster algebra; also known as \emph{cluster mutation}. Finally, after providing an example of these different concepts, we state a very important result within Cluster algebras, known as the \emph{Positivity theorem}, or \emph{Positivity conjecture} before it was proven in \cite{LS} for every \emph{skew-symmetric} cluster algebra; and in \cite{GHKK} for the general case of \emph{skew-symmetrizable} cluster algebras.

In chapter 2, we delve deeper into cluster algebras that are of \emph{surface type}; i.e. associated to a pair $(S,M)$, with $S$ a surface and $M$ a set of marked points on the boundary components of $S$. After constructing a \emph{triangulation} $T$ for $(S,M)$, we can consider an \emph{arc} $\gamma$ between two marked points. To this arc, we can then assign a \emph{snake graph}; which will be useful in later chapters.

In chapter 3, we move to a more number theoretic perspective and begin by explaining the idea of \emph{continued fractions}. To each continued fraction $[a_1,\dots,a_n]$, the authors in \cite{CS1} described a way to assign a \emph{sign function}, which then can be used to construct a corresponding snake graph. Consequently, we will describe the idea of the \emph{palindromification} $[a_n,\dots,a_1,a_1,\dots,a_n]$ of a continued fraction; and provide a few results about it to then apply to the theory of Markov numbers. 

Finally, in chapter 4, we state Frobenius' conjecture; namely that every \emph{Markov triple} (a solution to Markov's equation) is uniquely determined by its largest element. Moreover, via the triangulated punctured torus, we connect the solutions to Markov's equation to the idea of cluster algebras; more precisely, by using mutations, we construct a map that takes a Markov triple and outputs another Markov triple. 

By examining slopes $p/q$, with $p<q$ and gcd$(p,q) = 1$ in the natural number lattice $\mathbb{N}\times \mathbb{N}$, we can assign to each a Markov number denoted $m_{p/q}$. This will be done by constructing a snake graph corresponding to the slope (via its \emph{Christoffel path}); and calculating the number of perfect matching; which Frobenius showed always is a Markov number. This will then give us the necessary tools to find a reformulation for Frobenius' conjecture in purely cluster algebraic terms.

In the last part of chapter 5, by using \emph{Skein relations} we describe the concepts of a \emph{left} and \emph{right deformation} of an arc $\gamma$ between two lattice points in $\mathbb{N}\times \mathbb{N}$. Consequently, we will then provide some results which will ultimately be useful to prove a conjecture posed by Martin Aigner on the ordering of Markov numbers in \cite{A}.



\chapter{Introduction to Cluster algebras}
The definition of a Cluster algebra is not particularly difficult; however, it is involved. We begin by describing the space we are examining. Let $(\mathcal{G}, \cdot)$ be any free abelian (multiplicative) group, with basis $\mathbf{y} = \{y_1,\dots,y_n\}$. Next, define an operation $\oplus$ by;
\begin{equation}
    \prod_j y_j^{a_j} \oplus \prod_j y_j^{b_j} = \prod_j y_j^{(\text{min}(a_j,b_j))};  
\end{equation}
e.g. $y_1^3y_2^{-4}y_3y_4^5 \oplus y_1^{-1} = y_1^{-1}y_2^{-4}$. The reader may like to verify that $\oplus$ is indeed well-defined. Finally, we obtain that $(\mathcal{G},\oplus, \cdot)$ is semifield, i.e. $\oplus$ is commutative, associative and distributive with respect to multiplication in $\mathcal{G}$; more precisely, due to the nature of the operation $\oplus$, $(\mathcal{G},\oplus, \cdot)$ is also known as a \emph{tropical semifield}. Finally, consider the group ring $\mathbb{Z}\mathcal{G}$ of $\mathcal{G}$ and note that $\mathbb{Z}\mathcal{G}$ is exactly the ring of \emph{Laurent polynomials} in the variables $y_1,\dots, y_n$; this will be the used as the ground ring for the corresponding Cluster algebra. 
\section{Quivers, initial seeds and mutations}
A \emph{quiver} $\mathcal{Q}$ is a directed graph; i.e. a $4$-tuple $(\mathcal{Q}_0, \mathcal{Q}_1, h, t)$, where $\mathcal{Q}_0$ and $\mathcal{Q}_1$ are the collections of vertices and arrows, respectively. Similarly, $h,t: \mathcal{Q}_1 \to \mathcal{Q}_0$, are set functions that map the head and tail of each arrow in $\mathcal{Q}_1$ in the appropriate direction. Moreover, if $\mathcal{Q}$ does not have any 2-cycles, i.e. $\circ \rightleftarrows \circ$, and does not have any loop, which is simply an arrow from a vertex to itself, then $\mathcal{Q}$ is called a \emph{Cluster quiver}. These will be our main focus throughout the paper.
\begin{figure}[!htb]
\centering
\minipage[c]{0.32\textwidth}
\footnotesize
  \[\begin{tikzcd}
	&& 1 \\
	\\
	2 &&&& 3 \\
	\\
	&& 4
	\arrow[from=5-3, to=1-3]
	\arrow[shift left=2, from=3-1, to=5-3]
	\arrow[shift left=2, from=5-3, to=3-5]
	\arrow[from=1-3, to=3-5]
	\arrow[from=1-3, to=3-1]
	\arrow[shift right=1, from=5-3, to=3-5]
	\arrow[from=3-1, to=5-3]
	\arrow[shift right=2, from=3-1, to=5-3]
\end{tikzcd}\]
\endminipage\hfill
\minipage[c]{0.32\textwidth}
\footnotesize
  \[\begin{tikzcd}
	&& 1 \\
	\\
	2 &&&& 3
	\arrow[from=1-3, to=3-5]
	\arrow[shift right=1, from=1-3, to=3-1]
	\arrow[shift right=1, from=3-1, to=1-3]
	\arrow[from=3-1, to=3-5]
	\arrow[shift right=2, from=3-1, to=3-5]
    \arrow[loop right, to = 5-3]
\end{tikzcd}\]
\endminipage\hfill
\minipage[c]{0.32\textwidth}%
\footnotesize
\[\begin{tikzcd}
	1 && 2 && 3 \\
	&& 4
	\arrow[from=1-1, to=1-3]
	\arrow[from=1-5, to=1-3]
	\arrow[from=2-3, to=1-3]
	\arrow[curve={height=-30pt}, from=1-1, to=1-5]
\end{tikzcd}\]
\endminipage
\caption{Examples of \emph{Cluster quivers} (left and rightmost); and example of quiver that is not of the Cluster type (center); notice that it has a 2-cycle, namely $1 \rightleftarrows 2$, and has a loop. }
\end{figure}
\\
A \emph{seed} $(\mathbf{x},\mathbf{y},\mathcal{Q})$ is what determines the corresponding Cluster algebra, denoted $\mathcal{A} = \mathcal{A}(\mathbf{x},\mathbf{y},\mathcal{Q})$, (through a few rules); where
    \begin{itemize}
        \item $\mathbf{x}=(x_1,\dots,x_n)$ is the $n$-tuple of variables, called the \emph{initial Cluster}; e.g. in the setting of triangulated polygons, these are precisely the diagonals of the triangulation. 
        \item $\mathbf{y}=(y_1,\dots,y_n)$ is the $n$-tuple of generators, called the \emph{initial coefficients}; e.g. in the setting of Conway-Coxeter frieze patterns, all these variables are equal to 1; or rather, the (free abelian) group corresponding to Conway-Coxeter frieze patterns is precisely $\mathbb{Z}$, with the usual multiplication, which has basis $1$.
    	\item $\mathcal{Q}$ a Cluster quiver;
    \end{itemize}
The process of generating a Cluster algebra from an initial seed is by iterating what is called a \emph{Cluster mutation}, or simply \emph{mutation}. A mutation $\mu_k$ acts on the initial seed as follows; 

\begin{itemize}
    \item $\Tilde{\mathbf{x}} :=\mu_k\mathbf{x} = \{\mathbf{x} / x_k\} \cup \{x_k^{'}\}$; where, 
        \begin{equation}
            x_k^{'} = \dfrac{1}{y_k \oplus 1}\dfrac{y_k\prod_{i \to k}x_i + \prod_{k \to j}x_j}{x_k};
        \end{equation}
        where the first product is over all arrows going into vertex $k$, in the corresponding quiver; and similarly, the second product is over all arrows going out from vertex $k$.
    \item $\Tilde{\mathbf{y}} := \mu_k\mathbf{y} = (y_1^{'},\dots,y_n^{'})$; where,
    \begin{equation*}
    y_j = 
        \begin{cases}
        y_j \prod_{k \to i}y_k(y_k \oplus 1)^{-1} \prod_{j \to k}(y_k \oplus 1),  \ &\text{if} \ j \ne k \\
        y_k^{-1}, &\text{if} \ j = k;
        \end{cases}
    \end{equation*}
    \item Lastly, $\mu_k$ acts on the quiver $\mathcal{Q}$ in the following way;
    \begin{itemize}
        \item[i.] For any path $i \to k \to j$, add an arrow $i \to j$,
        \item[ii.] Invert all arrows going into and coming out from vertex $k$,
        \item[iii.] Remove any $2$-cycles;
    \end{itemize}
    through this, we obtain a new quiver $\mathcal{Q}^{'}$.
\end{itemize}
Hence, we finally obtain that the new seed after mutation $\mu_k$ is precisely $(\Tilde{\mathbf{x}},\Tilde{\mathbf{y}}, \mathcal{Q}^{'})$. The Cluster algebra is then the  $\mathbb{F}\mathcal{G}$-subalgebra of $\mathcal{F}:= \mathbb{Q}\mathcal{G}(x_1,\dots,x_n)$; i.e. it is a subalgebra of the field of rational functions in $n$ variables and coefficients in $\mathbb{Z}\mathcal{G}$; generated by, what are called, \emph{Cluster variables}, which are obtained from the initial seed by a recursively applying mutations; the reader may quickly notice that given any seed there are $n$ different mutations $\mu_1, . . . , \mu_n$, one for each vertex of the quiver, which corresponds to the number of Cluster variables. We denote by $\mathcal{X}$, the set of all possible Cluster variables after any arbitrarily long sequence of mutations.\footnote[1]{Notice that mutations are involutions; i.e. $\mu_k\circ\mu_k =\textit{Id}$.}
\begin{figure}[!htb]
\centering
\minipage[c]{0.45\textwidth}
\footnotesize
  \[\begin{tikzcd}
	1 &&&& 2 \\
	& 3 \\
	&&&& 4 \\
	&&& 5
	\arrow[from=2-2, to=4-4]
	\arrow[from=2-2, to=1-5]
	\arrow[curve={height=-46pt}, from=1-5, to=4-4]
	\arrow[curve={height=12pt}, from=1-1, to=4-4]
	\arrow[shift right=1, from=1-5, to=3-5]
	\arrow[shift left=1, from=1-5, to=3-5]
	\arrow[from=1-1, to=2-2]
	\arrow[curve={height=-12pt}, from=1-1, to=1-5]
	\arrow[shift left=1, curve={height=-6pt}, from=4-4, to=1-5]
	\arrow[shift right=2, curve={height=12pt}, from=1-1, to=4-4]
    \end{tikzcd}\]
\endminipage\hfill
\minipage[c]{0.1\textwidth}
\Large
$\xleftrightarrow{\mu_3}$
\endminipage\hfill
\minipage[c]{0.45\textwidth}%
\footnotesize
\[\begin{tikzcd}
	1 &&&& 2 \\
	& 3 \\
	&&&& 4 \\
	&&& 5
	\arrow[curve={height=12pt}, from=1-1, to=4-4]
	\arrow[curve={height=-12pt}, from=1-1, to=1-5]
	\arrow[from=2-2, to=1-1]
	\arrow[from=1-5, to=2-2]
	\arrow[from=4-4, to=2-2]
	\arrow[curve={height=-6pt}, from=4-4, to=1-5]
	\arrow[shift left=1, from=1-5, to=3-5]
	\arrow[shift right=1, from=1-5, to=3-5]
	\arrow[curve={height=-46pt}, from=1-5, to=4-4]
	\arrow[shift right=2, curve={height=12pt}, from=1-1, to=4-4]
	\arrow[shift right=4, curve={height=12pt}, from=1-1, to=4-4]
	\arrow[shift left=2, curve={height=-12pt}, from=1-1, to=1-5]
\end{tikzcd}\]
\endminipage
\caption{Example of a quiver mutation $\mu_3$.}
\end{figure}
\begin{example}
Let the initial seed be 
\begin{equation*}
(\mathbf{x},\mathbf{y},\mathcal{Q}) = ((x_1,x_2,x_3),(1,1,1),\begin{tikzcd}
	1 & 2 & 3
	\arrow[from=1-1, to=1-2]
	\arrow[from=1-3, to=1-2]
	\arrow[curve={height=-12pt}, from=1-1, to=1-3].
\end{tikzcd})
\end{equation*}
Note that any mutation on $\mathbf{y}$ leaves it unchanged as we set it to be a vector of $1$'s; hence we will leave it out. Next, consider $\mu_1$, and we then obtain $x_1^{'} = (1+x_2x_3)/x_1$, and the resulting seed is now;
\begin{equation*}
    \left(\dfrac{1+x_2x_3}{x_1},x_2,x_3\right), \begin{tikzcd}
	1 & 2 & 3
	\arrow[from=1-2, to=1-1]
	\arrow[from=1-3, to=1-2]
	\arrow[curve={height=12pt}, from=1-3, to=1-1]
\end{tikzcd}.
\end{equation*}
We then apply mutation $\mu_2$, which yields;
\begin{equation*}
    \left(\dfrac{1 + x_2x_3}{x_1},\dfrac{x_1x_3+x_3x_2}{x_1x_2},x_3\right), \begin{tikzcd}
	1 & 2 & 3
	\arrow[from=1-1, to=1-2]
	\arrow[from=1-2, to=1-3]
	\arrow[curve={height=12pt}, from=1-3, to=1-1]
	\arrow[shift right=1, curve={height=12pt}, from=1-3, to=1-1]
\end{tikzcd}.
\end{equation*}
Apply mutation $\mu_3$;
\begin{equation*}
    \left(\dfrac{1 + x_2x_3}{x_1},\dfrac{1 + x_1x_3+x_3x_2}{x_1x_2},\dfrac{x_1 + x_2 + (x_1^2 + x_1x_2 + 2x_2^2)x_3 + x_2^3x_3^2}{x_1^2x_2x_3}\right),\begin{tikzcd}
	1 & 2 & 3
	\arrow[from=1-1, to=1-2]
	\arrow[from=1-3, to=1-2]
	\arrow[curve={height=-12pt}, from=1-1, to=1-3]
	\arrow[shift left=1, curve={height=-12pt}, from=1-1, to=1-3]
\end{tikzcd}.
\end{equation*}
Finally, we apply mutation $\mu_2$ once more;
\begin{equation*}
    \left(\dfrac{1 + x_2x_3}{x_1},\dfrac{x_1 + x_2 + x_2^2x_3}{x_1x_3},\dfrac{x_1 + x_2 + (x_1^2 + x_1x_2 + 2x_2^2)x_3 + x_2^3x_3^2}{x_1^2x_2x_3}\right),\begin{tikzcd}
	1 & 2 & 3
	\arrow[from=1-2, to=1-1]
	\arrow[from=1-3, to=1-2]
	\arrow[curve={height=-12pt}, from=1-1, to=1-3]
	\arrow[shift left=1, curve={height=-12pt}, from=1-1, to=1-3]
\end{tikzcd}.
\end{equation*}
In general, this process of mutating will always yield a new unseen Cluster variable; i.e., in general, there are infinitely many Cluster variable that arise from Cluster mutations of an arbitrary seed. One may verify that for an initial seed $((x_1,x_2),(1,1),1 \to 2)$, we obtain finitely many Cluster variables; more precisely, there are exactly 5. 
\end{example}
In the above example, we can observe that after every mutation, the Cluster variable obtained is, what is called, a \emph{Laurent polynomial}. This is no coincidence; and the following theorem does, in fact, generalize it;
\begin{theorem}
Let $c$ in $\mathcal{X}$ be any Cluster variable. Then,
\begin{equation*}
    c = \dfrac{f(x_1,\dots,x_n)}{x_1^{\tau_1}\cdots x_n^{\tau_n}};
\end{equation*}
with $f \in \mathbb{Z}\mathcal{G}[x_1,\dots,x_n]$.
\end{theorem}
This result is particularly surprising as \emph{a priori} any Cluster variable is simply a rational polynomial in the variables $x_1,\dots,x_n$; while in other settings, when dividing two seemingly unrelated multinomials, we in general are not able to simplify it so that we have a monomial in the denominator. 
\begin{remark}
The above result can be strengthened further by changing $\mathbb{Z}\mathcal{G}[x_1,\dots,x_n]$ to $\mathbb{Z}_{\geq 0}\mathcal{G}[x_1,\dots,x_n]$, also known as the \emph{positivity conjecture}; in other words, all coefficients of $f$, above, are positive. In \cite{LS}, the authors finally prove the conjecture for all \emph{skew-symmetric} cluster algebras. 
\end{remark}

\chapter{Cluster algebras of surface type}
Before we begin, we define what we mean by \emph{surface}, or of \emph{ surface type};
\begin{definition}
A \emph{surface} $(S,M)$ is a connected oriented Riemann surface $S$ with (possibly empty) boundary $\partial S$; together with a finite collection $M \subset S$ of \emph{marked points}; with the only condition that each boundary component $\mathbf{must}$ contain at least one marked point.
\end{definition}
\begin{figure}[H]
    \centering
    

\tikzset{every picture/.style={line width=0.75pt}} %set default line width to 0.75pt        

\begin{tikzpicture}[x=0.75pt,y=0.75pt,yscale=-1,xscale=1]
%uncomment if require: \path (0,471); %set diagram left start at 0, and has height of 471

%Shape: Ellipse [id:dp5936664030318399] 
\draw   (99,83.33) .. controls (99,77.07) and (117.51,72) .. (140.33,72) .. controls (163.16,72) and (181.67,77.07) .. (181.67,83.33) .. controls (181.67,89.59) and (163.16,94.67) .. (140.33,94.67) .. controls (117.51,94.67) and (99,89.59) .. (99,83.33) -- cycle ;
%Shape: Circle [id:dp886964729900558] 
\draw   (99,82.33) .. controls (99,59.51) and (117.51,41) .. (140.33,41) .. controls (163.16,41) and (181.67,59.51) .. (181.67,82.33) .. controls (181.67,105.16) and (163.16,123.67) .. (140.33,123.67) .. controls (117.51,123.67) and (99,105.16) .. (99,82.33) -- cycle ;
%Shape: Circle [id:dp9505094746069663] 
\draw   (325.8,78.33) .. controls (325.8,55.51) and (344.31,37) .. (367.13,37) .. controls (389.96,37) and (408.47,55.51) .. (408.47,78.33) .. controls (408.47,101.16) and (389.96,119.67) .. (367.13,119.67) .. controls (344.31,119.67) and (325.8,101.16) .. (325.8,78.33) -- cycle ;
%Shape: Circle [id:dp0899912035587026] 
\draw  [fill={rgb, 255:red, 15; green, 1; blue, 1 }  ,fill opacity=1 ] (117.83,64.35) .. controls (117.83,63.47) and (118.55,62.77) .. (119.43,62.78) .. controls (120.31,62.78) and (121.01,63.5) .. (121,64.38) .. controls (120.99,65.26) and (120.27,65.96) .. (119.4,65.95) .. controls (118.52,65.94) and (117.82,65.23) .. (117.83,64.35) -- cycle ;
%Shape: Circle [id:dp6344129113519207] 
\draw  [fill={rgb, 255:red, 15; green, 1; blue, 1 }  ,fill opacity=1 ] (152.63,81.55) .. controls (152.63,80.67) and (153.35,79.97) .. (154.23,79.98) .. controls (155.11,79.98) and (155.81,80.7) .. (155.8,81.58) .. controls (155.79,82.46) and (155.07,83.16) .. (154.2,83.15) .. controls (153.32,83.14) and (152.62,82.43) .. (152.63,81.55) -- cycle ;
%Shape: Circle [id:dp4511676376300491] 
\draw  [fill={rgb, 255:red, 15; green, 1; blue, 1 }  ,fill opacity=1 ] (122.23,111.15) .. controls (122.23,110.27) and (122.95,109.57) .. (123.83,109.58) .. controls (124.71,109.58) and (125.41,110.3) .. (125.4,111.18) .. controls (125.39,112.06) and (124.67,112.76) .. (123.8,112.75) .. controls (122.92,112.74) and (122.22,112.03) .. (122.23,111.15) -- cycle ;
%Shape: Circle [id:dp4148375320769103] 
\draw  [fill={rgb, 255:red, 15; green, 1; blue, 1 }  ,fill opacity=1 ] (161.83,102.35) .. controls (161.83,101.47) and (162.55,100.77) .. (163.43,100.78) .. controls (164.31,100.78) and (165.01,101.5) .. (165,102.38) .. controls (164.99,103.26) and (164.27,103.96) .. (163.4,103.95) .. controls (162.52,103.94) and (161.82,103.23) .. (161.83,102.35) -- cycle ;
%Shape: Circle [id:dp13358398817698114] 
\draw  [fill={rgb, 255:red, 228; green, 227; blue, 188 }  ,fill opacity=1 ] (346.47,78.33) .. controls (346.47,66.92) and (355.72,57.67) .. (367.13,57.67) .. controls (378.55,57.67) and (387.8,66.92) .. (387.8,78.33) .. controls (387.8,89.75) and (378.55,99) .. (367.13,99) .. controls (355.72,99) and (346.47,89.75) .. (346.47,78.33) -- cycle ;
%Shape: Circle [id:dp8855500282436761] 
\draw  [fill={rgb, 255:red, 15; green, 1; blue, 1 }  ,fill opacity=1 ] (326.63,91.55) .. controls (326.63,90.67) and (327.35,89.97) .. (328.23,89.98) .. controls (329.11,89.98) and (329.81,90.7) .. (329.8,91.58) .. controls (329.79,92.46) and (329.07,93.16) .. (328.2,93.15) .. controls (327.32,93.14) and (326.62,92.43) .. (326.63,91.55) -- cycle ;
%Shape: Circle [id:dp9212146321694251] 
\draw  [fill={rgb, 255:red, 15; green, 1; blue, 1 }  ,fill opacity=1 ] (365.53,99.57) .. controls (365.54,98.7) and (366.26,97.99) .. (367.13,98) .. controls (368.01,98.01) and (368.71,98.73) .. (368.71,99.6) .. controls (368.7,100.48) and (367.98,101.18) .. (367.1,101.18) .. controls (366.23,101.17) and (365.52,100.45) .. (365.53,99.57) -- cycle ;
%Shape: Circle [id:dp9296470673413565] 
\draw  [fill={rgb, 255:red, 15; green, 1; blue, 1 }  ,fill opacity=1 ] (405.13,66.77) .. controls (405.14,65.9) and (405.86,65.19) .. (406.73,65.2) .. controls (407.61,65.21) and (408.31,65.93) .. (408.31,66.8) .. controls (408.3,67.68) and (407.58,68.38) .. (406.7,68.38) .. controls (405.83,68.37) and (405.12,67.65) .. (405.13,66.77) -- cycle ;
%Shape: Circle [id:dp9075391788904883] 
\draw  [fill={rgb, 255:red, 15; green, 1; blue, 1 }  ,fill opacity=1 ] (357.53,48.37) .. controls (357.54,47.5) and (358.26,46.79) .. (359.13,46.8) .. controls (360.01,46.81) and (360.71,47.53) .. (360.71,48.4) .. controls (360.7,49.28) and (359.98,49.98) .. (359.1,49.98) .. controls (358.23,49.97) and (357.52,49.25) .. (357.53,48.37) -- cycle ;
%Shape: Ellipse [id:dp4018320458815504] 
\draw   (208,83.17) .. controls (208,68.72) and (228.51,57) .. (253.8,57) .. controls (279.09,57) and (299.6,68.72) .. (299.6,83.17) .. controls (299.6,97.63) and (279.09,109.34) .. (253.8,109.34) .. controls (228.51,109.34) and (208,97.63) .. (208,83.17) -- cycle ;
%Curve Lines [id:da03186435929475773] 
\draw    (240.27,82.17) .. controls (250.27,91.5) and (262.93,87.5) .. (267.6,81.5) ;
%Curve Lines [id:da7981103996078125] 
\draw    (244.93,85.5) .. controls (250.27,78.17) and (254.93,76.83) .. (262.27,86.17) ;
%Curve Lines [id:da8812010320833625] 
\draw    (462.18,76.07) .. controls (473.31,86.46) and (487.41,82.01) .. (492.6,75.33) ;
%Curve Lines [id:da6506246829588368] 
\draw    (467.37,79.78) .. controls (473.31,71.62) and (478.5,70.14) .. (486.66,80.52) ;
%Shape: Ellipse [id:dp9250644288260443] 
\draw  [fill={rgb, 255:red, 228; green, 227; blue, 188 }  ,fill opacity=1 ] (437.59,73.51) .. controls (440.4,64) and (445.08,56.29) .. (448.03,56.29) .. controls (450.99,56.29) and (451.11,64) .. (448.3,73.51) .. controls (445.49,83.02) and (440.81,90.74) .. (437.86,90.74) .. controls (434.9,90.74) and (434.78,83.02) .. (437.59,73.51) -- cycle ;
%Shape: Ellipse [id:dp5372444143214322] 
\draw  [fill={rgb, 255:red, 228; green, 227; blue, 188 }  ,fill opacity=1 ] (505.85,72.77) .. controls (503.34,63.26) and (503.69,55.55) .. (506.65,55.55) .. controls (509.61,55.55) and (514.04,63.26) .. (516.56,72.77) .. controls (519.07,82.28) and (518.71,89.99) .. (515.76,89.99) .. controls (512.8,89.99) and (508.37,82.28) .. (505.85,72.77) -- cycle ;
%Curve Lines [id:da07938845410908513] 
\draw    (437.86,90.74) .. controls (460.62,101.45) and (482.44,107.23) .. (515.76,89.99) ;
%Curve Lines [id:da039009680495956545] 
\draw    (448.03,56.29) .. controls (479.76,32.89) and (464.63,76.07) .. (506.65,55.55) ;
%Shape: Circle [id:dp3402244483606086] 
\draw  [fill={rgb, 255:red, 15; green, 1; blue, 1 }  ,fill opacity=1 ] (251.53,79.57) .. controls (251.54,78.7) and (252.26,77.99) .. (253.13,78) .. controls (254.01,78.01) and (254.71,78.73) .. (254.71,79.6) .. controls (254.7,80.48) and (253.98,81.18) .. (253.1,81.18) .. controls (252.23,81.17) and (251.52,80.45) .. (251.53,79.57) -- cycle ;
%Shape: Circle [id:dp1515362552946773] 
\draw  [fill={rgb, 255:red, 15; green, 1; blue, 1 }  ,fill opacity=1 ] (225.13,104.37) .. controls (225.14,103.5) and (225.86,102.79) .. (226.73,102.8) .. controls (227.61,102.81) and (228.31,103.53) .. (228.31,104.4) .. controls (228.3,105.28) and (227.58,105.98) .. (226.7,105.98) .. controls (225.83,105.97) and (225.12,105.25) .. (225.13,104.37) -- cycle ;
%Shape: Circle [id:dp540351092308096] 
\draw  [fill={rgb, 255:red, 15; green, 1; blue, 1 }  ,fill opacity=1 ] (275.93,92.37) .. controls (275.94,91.5) and (276.66,90.79) .. (277.53,90.8) .. controls (278.41,90.81) and (279.11,91.53) .. (279.11,92.4) .. controls (279.1,93.28) and (278.38,93.98) .. (277.5,93.98) .. controls (276.63,93.97) and (275.92,93.25) .. (275.93,92.37) -- cycle ;
%Shape: Circle [id:dp056198737530079645] 
\draw  [fill={rgb, 255:red, 15; green, 1; blue, 1 }  ,fill opacity=1 ] (235.13,58.77) .. controls (235.14,57.9) and (235.86,57.19) .. (236.73,57.2) .. controls (237.61,57.21) and (238.31,57.93) .. (238.31,58.8) .. controls (238.3,59.68) and (237.58,60.38) .. (236.7,60.38) .. controls (235.83,60.37) and (235.12,59.65) .. (235.13,58.77) -- cycle ;
%Shape: Circle [id:dp4449482163456263] 
\draw  [fill={rgb, 255:red, 15; green, 1; blue, 1 }  ,fill opacity=1 ] (294.73,73.57) .. controls (294.74,72.7) and (295.46,71.99) .. (296.33,72) .. controls (297.21,72.01) and (297.91,72.73) .. (297.91,73.6) .. controls (297.9,74.48) and (297.18,75.18) .. (296.3,75.18) .. controls (295.43,75.17) and (294.72,74.45) .. (294.73,73.57) -- cycle ;
%Shape: Circle [id:dp5622367823369225] 
\draw  [fill={rgb, 255:red, 15; green, 1; blue, 1 }  ,fill opacity=1 ] (220.33,79.97) .. controls (220.34,79.1) and (221.06,78.39) .. (221.93,78.4) .. controls (222.81,78.41) and (223.51,79.13) .. (223.51,80) .. controls (223.5,80.88) and (222.78,81.58) .. (221.9,81.58) .. controls (221.03,81.57) and (220.32,80.85) .. (220.33,79.97) -- cycle ;
%Shape: Ellipse [id:dp8536461228034546] 
\draw  [fill={rgb, 255:red, 15; green, 1; blue, 1 }  ,fill opacity=1 ] (446.51,75.26) .. controls (446.52,74.29) and (447.32,73.5) .. (448.3,73.51) .. controls (449.27,73.52) and (450.06,74.32) .. (450.05,75.3) .. controls (450.04,76.27) and (449.24,77.05) .. (448.26,77.05) .. controls (447.29,77.04) and (446.5,76.24) .. (446.51,75.26) -- cycle ;
%Shape: Ellipse [id:dp15602120118848628] 
\draw  [fill={rgb, 255:red, 15; green, 1; blue, 1 }  ,fill opacity=1 ] (470.11,100.19) .. controls (470.12,99.21) and (470.92,98.43) .. (471.89,98.44) .. controls (472.87,98.45) and (473.65,99.25) .. (473.64,100.22) .. controls (473.63,101.2) and (472.83,101.98) .. (471.86,101.97) .. controls (470.88,101.96) and (470.1,101.17) .. (470.11,100.19) -- cycle ;
%Shape: Ellipse [id:dp7286045659331059] 
\draw  [fill={rgb, 255:red, 15; green, 1; blue, 1 }  ,fill opacity=1 ] (478.12,73.93) .. controls (478.13,72.95) and (478.93,72.17) .. (479.9,72.18) .. controls (480.88,72.19) and (481.66,72.98) .. (481.65,73.96) .. controls (481.64,74.94) and (480.85,75.72) .. (479.87,75.71) .. controls (478.89,75.7) and (478.11,74.9) .. (478.12,73.93) -- cycle ;
%Shape: Ellipse [id:dp9636443945093016] 
\draw  [fill={rgb, 255:red, 15; green, 1; blue, 1 }  ,fill opacity=1 ] (499.49,87.73) .. controls (499.5,86.75) and (500.3,85.97) .. (501.27,85.98) .. controls (502.25,85.99) and (503.03,86.78) .. (503.02,87.76) .. controls (503.01,88.74) and (502.21,89.52) .. (501.24,89.51) .. controls (500.26,89.5) and (499.48,88.7) .. (499.49,87.73) -- cycle ;




\end{tikzpicture}

    \caption{Examples of marked surfaces, with 0, 0, 1, and 2 boundary components, respectively.}
\end{figure}
Any marked point $p \in M$ such that $p \notin \partial S$ is referred to as a \emph{puncture}.
Once we have a surface, we define additional structures.\begin{definition}(\emph{Ordinary arcs}) 
    An \emph{arc} $\gamma$ in $(S,M)$ is a curve in $S$, considered up to isotopy, such that;
    \begin{itemize}
        \item Its endpoints are in $M$,
        \item It does not intersect itself, except for possibly having overlapping endpoints,
        \item Besides its endpoints, $\gamma$ is disjoint from $M$,
        \item $\gamma$ does not cut out an unpunctured monogon or bigon.
    \end{itemize}
\end{definition}
For the sake of labeling, an arc that starts and ends in the same points is called a \emph{loop}.
Moreover, suppose we have two arcs $\gamma, \tilde{\gamma}$ on a surface $S$, then we define $e(\gamma, \tilde{\gamma})$ to be the number of intersections between the two arcs when considering all possible isotopy\footnote{An \emph{isotopy} is a homotopy $h: [0,1]\times X \to Y$ such that for each $t \in [0,1]$, $h(t,\bullet)$ is a homeomorphism.} equivalent arcs for both. If $e(\gamma,\tilde{\gamma}) = 0$, then we say that the arcs $\gamma$ and $\tilde{\gamma}$ are compatible. If a maximal triangulation $T$ consists entirely of pairwise compatible arcs, then it is called an \emph{ideal triangulation}, which each triangular region called an \emph{ideal triangle}.
\begin{figure}[H]
    \centering
    

\tikzset{every picture/.style={line width=0.75pt}} %set default line width to 0.75pt        

\begin{tikzpicture}[x=0.75pt,y=0.75pt,yscale=-1,xscale=1]
%uncomment if require: \path (0,471); %set diagram left start at 0, and has height of 471

%Shape: Triangle [id:dp9282044160141008] 
\draw   (150.39,93.31) -- (187.77,160.18) -- (113.01,160.18) -- cycle ;
\draw  [fill={rgb, 255:red, 19; green, 1; blue, 1 }  ,fill opacity=1 ] (110.92,159.64) .. controls (110.92,158.05) and (112.2,156.77) .. (113.78,156.77) .. controls (115.36,156.77) and (116.64,158.05) .. (116.64,159.64) .. controls (116.64,161.22) and (115.36,162.5) .. (113.78,162.5) .. controls (112.2,162.5) and (110.92,161.22) .. (110.92,159.64) -- cycle ; \draw   (110.92,159.64) -- (116.64,159.64) ; \draw   (113.78,156.77) -- (113.78,162.5) ;
\draw  [fill={rgb, 255:red, 19; green, 1; blue, 1 }  ,fill opacity=1 ] (183.82,159.64) .. controls (183.82,158.05) and (185.1,156.77) .. (186.69,156.77) .. controls (188.27,156.77) and (189.55,158.05) .. (189.55,159.64) .. controls (189.55,161.22) and (188.27,162.5) .. (186.69,162.5) .. controls (185.1,162.5) and (183.82,161.22) .. (183.82,159.64) -- cycle ; \draw   (183.82,159.64) -- (189.55,159.64) ; \draw   (186.69,156.77) -- (186.69,162.5) ;
\draw  [fill={rgb, 255:red, 19; green, 1; blue, 1 }  ,fill opacity=1 ] (148.07,93.7) .. controls (148.07,92.12) and (149.35,90.83) .. (150.93,90.83) .. controls (152.51,90.83) and (153.79,92.12) .. (153.79,93.7) .. controls (153.79,95.28) and (152.51,96.56) .. (150.93,96.56) .. controls (149.35,96.56) and (148.07,95.28) .. (148.07,93.7) -- cycle ; \draw   (148.07,93.7) -- (153.79,93.7) ; \draw   (150.93,90.83) -- (150.93,96.56) ;
%Shape: Tear Drop [id:dp6032198370858309] 
\draw   (234.8,144.12) .. controls (234.8,144.12) and (234.8,144.12) .. (234.8,144.12) .. controls (226.04,135.39) and (226.01,121.21) .. (234.74,112.45) .. controls (243.47,103.69) and (257.65,103.66) .. (266.41,112.39) .. controls (275.17,121.12) and (275.2,135.3) .. (266.47,144.06) .. controls (255.94,154.63) and (250.66,159.93) .. (250.66,159.93) .. controls (250.66,159.93) and (245.38,154.66) .. (234.8,144.12) -- cycle ;
%Shape: Ellipse [id:dp3071299319516321] 
\draw   (251.34,82.18) .. controls (268.87,82.16) and (283,99.54) .. (282.92,121.02) .. controls (282.83,142.49) and (268.55,159.92) .. (251.03,159.95) .. controls (233.5,159.98) and (219.36,142.59) .. (219.44,121.12) .. controls (219.53,99.64) and (233.81,82.21) .. (251.34,82.18) -- cycle ;
\draw  [fill={rgb, 255:red, 19; green, 1; blue, 1 }  ,fill opacity=1 ] (247.75,158.92) .. controls (247.75,157.21) and (249.13,155.83) .. (250.83,155.83) .. controls (252.54,155.83) and (253.92,157.21) .. (253.92,158.92) .. controls (253.92,160.62) and (252.54,162) .. (250.83,162) .. controls (249.13,162) and (247.75,160.62) .. (247.75,158.92) -- cycle ; \draw   (247.75,158.92) -- (253.92,158.92) ; \draw   (250.83,155.83) -- (250.83,162) ;
\draw  [fill={rgb, 255:red, 19; green, 1; blue, 1 }  ,fill opacity=1 ] (247.75,105.42) .. controls (247.75,103.71) and (249.13,102.33) .. (250.83,102.33) .. controls (252.54,102.33) and (253.92,103.71) .. (253.92,105.42) .. controls (253.92,107.12) and (252.54,108.5) .. (250.83,108.5) .. controls (249.13,108.5) and (247.75,107.12) .. (247.75,105.42) -- cycle ; \draw   (247.75,105.42) -- (253.92,105.42) ; \draw   (250.83,102.33) -- (250.83,108.5) ;
%Shape: Tear Drop [id:dp6386482616728619] 
\draw   (318.22,137.97) .. controls (318.22,137.97) and (318.22,137.97) .. (318.22,137.97) .. controls (318.22,137.97) and (318.22,137.97) .. (318.22,137.97) .. controls (306.18,126.59) and (305.65,107.6) .. (317.03,95.56) .. controls (328.41,83.52) and (347.4,82.98) .. (359.44,94.37) .. controls (371.48,105.75) and (372.02,124.74) .. (360.63,136.78) .. controls (360.63,136.78) and (360.63,136.78) .. (360.63,136.78) .. controls (346.89,151.3) and (340.03,158.57) .. (340.03,158.58) .. controls (340.03,158.57) and (332.76,151.7) .. (318.22,137.97) -- cycle ;
%Straight Lines [id:da4842260407951211] 
\draw    (340.03,158.58) -- (339.92,119.08) ;
\draw  [fill={rgb, 255:red, 19; green, 1; blue, 1 }  ,fill opacity=1 ] (337.25,158.08) .. controls (337.25,156.38) and (338.63,155) .. (340.33,155) .. controls (342.04,155) and (343.42,156.38) .. (343.42,158.08) .. controls (343.42,159.79) and (342.04,161.17) .. (340.33,161.17) .. controls (338.63,161.17) and (337.25,159.79) .. (337.25,158.08) -- cycle ; \draw   (337.25,158.08) -- (343.42,158.08) ; \draw   (340.33,155) -- (340.33,161.17) ;
\draw  [fill={rgb, 255:red, 192; green, 185; blue, 185 }  ,fill opacity=1 ] (336.75,118.58) .. controls (336.75,116.88) and (338.13,115.5) .. (339.83,115.5) .. controls (341.54,115.5) and (342.92,116.88) .. (342.92,118.58) .. controls (342.92,120.29) and (341.54,121.67) .. (339.83,121.67) .. controls (338.13,121.67) and (336.75,120.29) .. (336.75,118.58) -- cycle ; \draw   (336.75,118.58) -- (342.92,118.58) ; \draw   (339.83,115.5) -- (339.83,121.67) ;
%Shape: Ellipse [id:dp6950674907452231] 
\draw   (409.03,117.51) .. controls (412.32,116) and (419.33,124.27) .. (424.69,135.98) .. controls (430.04,147.68) and (431.71,158.39) .. (428.42,159.9) .. controls (425.13,161.41) and (418.12,153.14) .. (412.76,141.43) .. controls (407.41,129.73) and (405.74,119.01) .. (409.03,117.51) -- cycle ;
%Shape: Ellipse [id:dp09560336913224654] 
\draw   (445.33,114.53) .. controls (448.83,115.46) and (449.01,126.3) .. (445.72,138.74) .. controls (442.44,151.19) and (436.94,160.54) .. (433.44,159.61) .. controls (429.94,158.69) and (429.76,147.85) .. (433.04,135.4) .. controls (436.33,122.95) and (441.82,113.61) .. (445.33,114.53) -- cycle ;
\draw  [fill={rgb, 255:red, 19; green, 1; blue, 1 }  ,fill opacity=1 ] (427.51,159.15) .. controls (427.51,157.3) and (429.01,155.79) .. (430.86,155.79) .. controls (432.71,155.79) and (434.21,157.3) .. (434.21,159.15) .. controls (434.21,161) and (432.71,162.5) .. (430.86,162.5) .. controls (429.01,162.5) and (427.51,161) .. (427.51,159.15) -- cycle ; \draw   (427.51,159.15) -- (434.21,159.15) ; \draw   (430.86,155.79) -- (430.86,162.5) ;
%Shape: Ellipse [id:dp675764835830011] 
\draw   (429.78,92.75) .. controls (445.16,92.73) and (457.57,107.98) .. (457.49,126.83) .. controls (457.42,145.67) and (444.89,160.96) .. (429.51,160.99) .. controls (414.13,161.01) and (401.72,145.76) .. (401.8,126.91) .. controls (401.87,108.07) and (414.4,92.77) .. (429.78,92.75) -- cycle ;




\end{tikzpicture}

    \caption{Ordinary triangle, two-vertex triangle, self-folded triangle and one-vertex triangle.}
\end{figure}
In the figure above we see the four possible types of ideal triangles in a possible ideal triangulation. In this paper we will focus mostly on the first (and perhaps the fourth one too) as the other two require a marked point that is not on any boundary component; which we will often not consider for the sake of length and clarity. 
\begin{theorem}{\normalfont \cite{Sch}}
The number of arcs in an ideal triangulation is exactly
\begin{equation*}
    n = 6g + 3b + 3p + c - 6;
\end{equation*}
where $g$ is the genus of $S$, $b$ is the number of boundary components, $p$ is the number of punctures and $c = |M|-p$ is the number of marked points on $\partial S$.
\end{theorem}
\begin{remark}
Note that each ideal triangulation is connected to all other possible triangulations by a series of \emph{flips}; that is, replacing an arc $\gamma$ by another arc $\Tilde{\gamma}$, so to obtain our new triangulation $\Tilde{T} = (T\backslash\{\gamma\}) \cup \{\Tilde{\gamma}\}$, in the following way; 
\begin{figure}[H]
    \centering
    

\tikzset{every picture/.style={line width=0.75pt}} %set default line width to 0.75pt        

\begin{tikzpicture}[x=0.75pt,y=0.75pt,yscale=-1,xscale=1]
%uncomment if require: \path (0,342); %set diagram left start at 0, and has height of 342

%Shape: Parallelogram [id:dp9440132437357889] 
\draw   (106.2,82.67) -- (212.43,82.67) -- (166.9,169.39) -- (60.67,169.39) -- cycle ;
%Shape: Parallelogram [id:dp4450674539847944] 
\draw   (296.2,83.33) -- (402.43,83.33) -- (356.9,170.06) -- (250.67,170.06) -- cycle ;
%Curve Lines [id:da015686636387745367] 
\draw    (106.2,82.67) .. controls (95.28,133.39) and (163.28,119.39) .. (166.9,169.39) ;
%Curve Lines [id:da5146201674257586] 
\draw    (250.67,170.06) .. controls (329.28,156.06) and (344.61,95.39) .. (402.43,83.33) ;
%Curve Lines [id:da8815901966148098] 
\draw    (195.94,131.39) .. controls (242.8,118.19) and (214.52,144.85) .. (260.53,129.86) ;
\draw [shift={(261.94,129.39)}, rotate = 161.57] [color={rgb, 255:red, 0; green, 0; blue, 0 }  ][line width=0.75]    (10.93,-3.29) .. controls (6.95,-1.4) and (3.31,-0.3) .. (0,0) .. controls (3.31,0.3) and (6.95,1.4) .. (10.93,3.29)   ;

% Text Node
\draw (134.78,108.07) node [anchor=north west][inner sep=0.75pt]    {$\gamma $};
% Text Node
\draw (314.11,110.07) node [anchor=north west][inner sep=0.75pt]    {$\tilde{\gamma }$};
% Text Node
\draw (218.78,111.33) node [anchor=north west][inner sep=0.75pt]   [align=left] {{\footnotesize flip}};


\end{tikzpicture}

    \caption{Example of flip at arc $\gamma$.}
\end{figure}
\end{remark}


\section{Cluster algebras from surfaces}
Now that we outlined the necessary definition, we will define the cluster algebra associated to a surface.
Let $T = \{\tau_1, \dots, \tau_n\}$ be an ideal trangulation of a surface $(S,M)$, and $\mathcal{Q}_T$ a cluster quiver defined as follows.
The vertices of $\mathcal{Q}_T$ are in bijection with the arcs of $T$; i.e. $\tau_i \mapsto i$. The arrows of $\mathcal{Q}_T$ are determined in the following way; for any triangle $\Delta$ in $T$, we draw an arrow $i \to j$ if $\tau_i$ and $\tau_j$ are sides of $\Delta$ with $\tau_j$ following $\tau_i$ in the clockwise order.
\begin{figure}[H]
    \centering
    

\tikzset{every picture/.style={line width=0.75pt}} %set default line width to 0.75pt        

\begin{tikzpicture}[x=0.75pt,y=0.75pt,yscale=-1,xscale=1]
%uncomment if require: \path (0,458); %set diagram left start at 0, and has height of 458

%Shape: Circle [id:dp011167431945667827] 
\draw   (11,248.38) .. controls (11,170.39) and (74.22,107.17) .. (152.21,107.17) .. controls (230.2,107.17) and (293.42,170.39) .. (293.42,248.38) .. controls (293.42,326.36) and (230.2,389.58) .. (152.21,389.58) .. controls (74.22,389.58) and (11,326.36) .. (11,248.38) -- cycle ;
%Shape: Circle [id:dp7048127758853319] 
\draw  [fill={rgb, 255:red, 228; green, 227; blue, 188 }  ,fill opacity=1 ] (118.21,248.38) .. controls (118.21,229.6) and (133.43,214.38) .. (152.21,214.38) .. controls (170.99,214.38) and (186.21,229.6) .. (186.21,248.38) .. controls (186.21,267.15) and (170.99,282.38) .. (152.21,282.38) .. controls (133.43,282.38) and (118.21,267.15) .. (118.21,248.38) -- cycle ;
\draw  [fill={rgb, 255:red, 19; green, 1; blue, 1 }  ,fill opacity=1 ] (77.92,126.55) .. controls (77.92,124.97) and (79.2,123.69) .. (80.78,123.69) .. controls (82.36,123.69) and (83.64,124.97) .. (83.64,126.55) .. controls (83.64,128.13) and (82.36,129.42) .. (80.78,129.42) .. controls (79.2,129.42) and (77.92,128.13) .. (77.92,126.55) -- cycle ; \draw   (77.92,126.55) -- (83.64,126.55) ; \draw   (80.78,123.69) -- (80.78,129.42) ;
\draw  [fill={rgb, 255:red, 19; green, 1; blue, 1 }  ,fill opacity=1 ] (59.92,357.55) .. controls (59.92,355.97) and (61.2,354.69) .. (62.78,354.69) .. controls (64.36,354.69) and (65.64,355.97) .. (65.64,357.55) .. controls (65.64,359.13) and (64.36,360.42) .. (62.78,360.42) .. controls (61.2,360.42) and (59.92,359.13) .. (59.92,357.55) -- cycle ; \draw   (59.92,357.55) -- (65.64,357.55) ; \draw   (62.78,354.69) -- (62.78,360.42) ;
\draw  [fill={rgb, 255:red, 19; green, 1; blue, 1 }  ,fill opacity=1 ] (136.42,217.55) .. controls (136.42,215.97) and (137.7,214.69) .. (139.28,214.69) .. controls (140.86,214.69) and (142.14,215.97) .. (142.14,217.55) .. controls (142.14,219.13) and (140.86,220.42) .. (139.28,220.42) .. controls (137.7,220.42) and (136.42,219.13) .. (136.42,217.55) -- cycle ; \draw   (136.42,217.55) -- (142.14,217.55) ; \draw   (139.28,214.69) -- (139.28,220.42) ;
\draw  [fill={rgb, 255:red, 19; green, 1; blue, 1 }  ,fill opacity=1 ] (167.92,276.05) .. controls (167.92,274.47) and (169.2,273.19) .. (170.78,273.19) .. controls (172.36,273.19) and (173.64,274.47) .. (173.64,276.05) .. controls (173.64,277.63) and (172.36,278.92) .. (170.78,278.92) .. controls (169.2,278.92) and (167.92,277.63) .. (167.92,276.05) -- cycle ; \draw   (167.92,276.05) -- (173.64,276.05) ; \draw   (170.78,273.19) -- (170.78,278.92) ;
\draw  [fill={rgb, 255:red, 19; green, 1; blue, 1 }  ,fill opacity=1 ] (269.42,322.55) .. controls (269.42,320.97) and (270.7,319.69) .. (272.28,319.69) .. controls (273.86,319.69) and (275.14,320.97) .. (275.14,322.55) .. controls (275.14,324.13) and (273.86,325.42) .. (272.28,325.42) .. controls (270.7,325.42) and (269.42,324.13) .. (269.42,322.55) -- cycle ; \draw   (269.42,322.55) -- (275.14,322.55) ; \draw   (272.28,319.69) -- (272.28,325.42) ;
\draw  [fill={rgb, 255:red, 19; green, 1; blue, 1 }  ,fill opacity=1 ] (285.92,213.05) .. controls (285.92,211.47) and (287.2,210.19) .. (288.78,210.19) .. controls (290.36,210.19) and (291.64,211.47) .. (291.64,213.05) .. controls (291.64,214.63) and (290.36,215.92) .. (288.78,215.92) .. controls (287.2,215.92) and (285.92,214.63) .. (285.92,213.05) -- cycle ; \draw   (285.92,213.05) -- (291.64,213.05) ; \draw   (288.78,210.19) -- (288.78,215.92) ;
%Curve Lines [id:da8637706579523565] 
\draw    (80.75,127.42) .. controls (80.75,126.42) and (81.25,127.92) .. (139.25,216.92) ;
%Curve Lines [id:da8931038200838086] 
\draw    (80.75,127.42) .. controls (181.25,177.42) and (236.75,256.42) .. (171.25,276.42) ;
%Curve Lines [id:da4496396466165504] 
\draw    (171.25,276.42) .. controls (171.25,275.42) and (171.25,276.42) .. (272.75,321.92) ;
%Curve Lines [id:da8914629129152412] 
\draw    (171.25,276.42) .. controls (171.25,275.42) and (241.25,279.42) .. (288.25,212.92) ;
%Curve Lines [id:da9415141411706016] 
\draw    (139.25,216.92) .. controls (32.75,226.42) and (105.25,322.42) .. (272.75,321.92) ;
%Curve Lines [id:da7175871064348958] 
\draw    (139.25,216.92) .. controls (28.75,180.5) and (66.25,256.5) .. (62.75,357) ;
%Straight Lines [id:da08883384268274452] 
\draw    (415.47,189) -- (484.4,189.1) ;
\draw [shift={(413.47,189)}, rotate = 0.08] [color={rgb, 255:red, 0; green, 0; blue, 0 }  ][line width=0.75]    (10.93,-3.29) .. controls (6.95,-1.4) and (3.31,-0.3) .. (0,0) .. controls (3.31,0.3) and (6.95,1.4) .. (10.93,3.29)   ;
%Straight Lines [id:da8222561468381305] 
\draw    (384.98,237.31) -- (404,198.7) ;
\draw [shift={(384.1,239.1)}, rotate = 296.22] [color={rgb, 255:red, 0; green, 0; blue, 0 }  ][line width=0.75]    (10.93,-3.29) .. controls (6.95,-1.4) and (3.31,-0.3) .. (0,0) .. controls (3.31,0.3) and (6.95,1.4) .. (10.93,3.29)   ;
%Straight Lines [id:da4094285868006836] 
\draw    (482.1,299.09) -- (414.28,298.61) ;
\draw [shift={(484.1,299.1)}, rotate = 180.4] [color={rgb, 255:red, 0; green, 0; blue, 0 }  ][line width=0.75]    (10.93,-3.29) .. controls (6.95,-1.4) and (3.31,-0.3) .. (0,0) .. controls (3.31,0.3) and (6.95,1.4) .. (10.93,3.29)   ;
%Straight Lines [id:da646049799635397] 
\draw    (404.1,289.1) -- (384.52,251.08) ;
\draw [shift={(383.6,249.3)}, rotate = 62.75] [color={rgb, 255:red, 0; green, 0; blue, 0 }  ][line width=0.75]    (10.93,-3.29) .. controls (6.95,-1.4) and (3.31,-0.3) .. (0,0) .. controls (3.31,0.3) and (6.95,1.4) .. (10.93,3.29)   ;
%Straight Lines [id:da7199087896373981] 
\draw    (494.1,199.1) -- (513.21,237.31) ;
\draw [shift={(514.1,239.1)}, rotate = 243.43] [color={rgb, 255:red, 0; green, 0; blue, 0 }  ][line width=0.75]    (10.93,-3.29) .. controls (6.95,-1.4) and (3.31,-0.3) .. (0,0) .. controls (3.31,0.3) and (6.95,1.4) .. (10.93,3.29)   ;
%Straight Lines [id:da2644151150412464] 
\draw    (513.2,251.38) -- (494.1,289.1) ;
\draw [shift={(514.1,249.6)}, rotate = 116.85] [color={rgb, 255:red, 0; green, 0; blue, 0 }  ][line width=0.75]    (10.93,-3.29) .. controls (6.95,-1.4) and (3.31,-0.3) .. (0,0) .. controls (3.31,0.3) and (6.95,1.4) .. (10.93,3.29)   ;
%Curve Lines [id:da3563655424763953] 
\draw [color={rgb, 255:red, 255; green, 0; blue, 0 }  ,draw opacity=1 ] [dash pattern={on 4.5pt off 4.5pt}]  (241.8,207) .. controls (232.05,166.91) and (198.91,143.99) .. (159.7,151.92) ;
\draw [shift={(157.9,152.3)}, rotate = 347.32] [color={rgb, 255:red, 255; green, 0; blue, 0 }  ,draw opacity=1 ][line width=0.75]    (10.93,-3.29) .. controls (6.95,-1.4) and (3.31,-0.3) .. (0,0) .. controls (3.31,0.3) and (6.95,1.4) .. (10.93,3.29)   ;
%Curve Lines [id:da8508534327134905] 
\draw [color={rgb, 255:red, 255; green, 0; blue, 0 }  ,draw opacity=1 ] [dash pattern={on 4.5pt off 4.5pt}]  (191.9,292.3) .. controls (182.25,291.34) and (173.34,289.07) .. (157.46,294.3) ;
\draw [shift={(155.7,294.9)}, rotate = 340.56] [color={rgb, 255:red, 255; green, 0; blue, 0 }  ,draw opacity=1 ][line width=0.75]    (10.93,-3.29) .. controls (6.95,-1.4) and (3.31,-0.3) .. (0,0) .. controls (3.31,0.3) and (6.95,1.4) .. (10.93,3.29)   ;
%Curve Lines [id:da2595129660274895] 
\draw [color={rgb, 255:red, 255; green, 0; blue, 0 }  ,draw opacity=1 ] [dash pattern={on 4.5pt off 4.5pt}]  (258.8,298) .. controls (267.72,287.51) and (270.78,284.42) .. (265.25,254.19) ;
\draw [shift={(264.9,252.3)}, rotate = 79.38] [color={rgb, 255:red, 255; green, 0; blue, 0 }  ,draw opacity=1 ][line width=0.75]    (10.93,-3.29) .. controls (6.95,-1.4) and (3.31,-0.3) .. (0,0) .. controls (3.31,0.3) and (6.95,1.4) .. (10.93,3.29)   ;
%Curve Lines [id:da3406149837992146] 
\draw [color={rgb, 255:red, 255; green, 0; blue, 0 }  ,draw opacity=1 ] [dash pattern={on 4.5pt off 4.5pt}]  (67.9,290.3) .. controls (78.57,279.63) and (80.77,278.37) .. (99.15,280.13) ;
\draw [shift={(100.9,280.3)}, rotate = 185.71] [color={rgb, 255:red, 255; green, 0; blue, 0 }  ,draw opacity=1 ][line width=0.75]    (10.93,-3.29) .. controls (6.95,-1.4) and (3.31,-0.3) .. (0,0) .. controls (3.31,0.3) and (6.95,1.4) .. (10.93,3.29)   ;
%Curve Lines [id:da24051616191269032] 
\draw [color={rgb, 255:red, 255; green, 0; blue, 0 }  ,draw opacity=1 ] [dash pattern={on 4.5pt off 4.5pt}]  (108.7,161.9) .. controls (110.41,156.77) and (115.78,156.03) .. (119.8,155.29) ;
\draw [shift={(121.7,154.9)}, rotate = 165.96] [color={rgb, 255:red, 255; green, 0; blue, 0 }  ,draw opacity=1 ][line width=0.75]    (10.93,-3.29) .. controls (6.95,-1.4) and (3.31,-0.3) .. (0,0) .. controls (3.31,0.3) and (6.95,1.4) .. (10.93,3.29)   ;
%Curve Lines [id:da5474461226079983] 
\draw [color={rgb, 255:red, 255; green, 0; blue, 0 }  ,draw opacity=1 ] [dash pattern={on 4.5pt off 4.5pt}]  (105.9,177.3) .. controls (107.6,178.81) and (106.8,188.72) .. (105.02,199.09) ;
\draw [shift={(104.7,200.9)}, rotate = 280.3] [color={rgb, 255:red, 255; green, 0; blue, 0 }  ,draw opacity=1 ][line width=0.75]    (10.93,-3.29) .. controls (6.95,-1.4) and (3.31,-0.3) .. (0,0) .. controls (3.31,0.3) and (6.95,1.4) .. (10.93,3.29)   ;

% Text Node
\draw (151,158.4) node [anchor=north west][inner sep=0.75pt]  [font=\scriptsize]  {$\tau _{1}$};
% Text Node
\draw (245.5,232.4) node [anchor=north west][inner sep=0.75pt]  [font=\scriptsize]  {$\tau _{2}$};
% Text Node
\draw (244,294.9) node [anchor=north west][inner sep=0.75pt]  [font=\scriptsize]  {$\tau _{3}$};
% Text Node
\draw (151,308.4) node [anchor=north west][inner sep=0.75pt]  [font=\scriptsize]  {$\tau _{4}$};
% Text Node
\draw (47.5,278.4) node [anchor=north west][inner sep=0.75pt]  [font=\scriptsize]  {$\tau _{5}$};
% Text Node
\draw (84,152.4) node [anchor=north west][inner sep=0.75pt]  [font=\scriptsize]  {$\tau _{6}$};
% Text Node
\draw (371.5,235.3) node [anchor=north west][inner sep=0.75pt]    {$1$};
% Text Node
\draw (402,180.8) node [anchor=north west][inner sep=0.75pt]    {$2$};
% Text Node
\draw (487,181.8) node [anchor=north west][inner sep=0.75pt]    {$3$};
% Text Node
\draw (517,236.3) node [anchor=north west][inner sep=0.75pt]    {$4$};
% Text Node
\draw (488.5,291.3) node [anchor=north west][inner sep=0.75pt]    {$5$};
% Text Node
\draw (400.5,291.8) node [anchor=north west][inner sep=0.75pt]    {$6$};


\end{tikzpicture}

    \caption{Example of triangulated surface (with 2 boundary components, also known as an \emph{annulus}) and its corresponding quiver.}
\end{figure}
In order to define an initial seed, we set the initial cluster $\mathbf{x}_T = \{\tau_1,\dots,\tau_n\}$; and we set $\mathbf{y}_T = \{y_1,\dots,y_n\}$ to be the initial coefficients (vectors) generating the tropical semifield $\mathcal{G}$.
Then, the cluster algebra $\mathcal{A} = \mathcal{A}(\mathbf{x_T},\mathbf{y}_T,\mathcal{Q}_T)$ is called the \emph{cluster algebra associated to the surface $(S,M)$ with principal coefficients in $T$}.

\section{Snake graphs}
A \emph{snake graph} $\mathcal{S}$ is a graph consisting of \emph{tiles}. A \emph{tile} $G$ is a square graph whose sides are orthogonal to the fixed basis; which we consider to be the standard orthonormal basis of the plane. Each tile will be isomorphic in the sense that side lengths are all equal. 

\begin{figure}[H]\label{tile}
    \centering
\begin{tikzpicture}[x=1cm,y=1cm]
	\node [square={snake square}] at (0,0) (B0) {$G$};
	\node [text width=4em,font=\tiny,shift={(0,-0.25)},align=center] at (B0.south) {South};
	\node [text width=4em,font=\tiny,shift={(0,0)},align=left] at (B0.west) {West};
	\node [text width=4em,font=\tiny,shift={(0,+0.25)},align=center] at (B0.north) {North};
	\node [text width=4em,font=\tiny,shift={(0,0)},align=right] at (B0.east) {East};
\end{tikzpicture}    
\caption{A tile $G$ with sides labeled to denote the orientation}
\end{figure}

Then, the snake graph $\mathcal{S} = (G_1,\dots,G_d)$ is a connected graph consisting of $d$ tiles $G_1,\dots,G_d$, where the tiles $G_i$ and $G_{i+1}$ share exactly 1 edge, $e_i$; which is either the north or east edge of tile $G_i$.
Next, we define the \emph{sign function}
\begin{equation*}
    f: \{\text{edges of} \  \mathcal{S}\} \to \{+,-\}.
\end{equation*}
such that for each tile $G_i$ the following hold;
\begin{itemize}
    \item The north and west edge have the same sign,
    \item The south and east edge have the same sign,
    \item The sign on the south edge is different than the sign on the north edge.
\end{itemize}
The reader may wonder about the purpose of having two, \emph{a priori}, equal representations of a snake graph; however, in section 3, after introducing continued fractions, the sign function will be of particular convenience to us.
\begin{figure}[H]
    \centering
    
\begin{figure}[!htb]
\centering
\minipage[c]{0.45\textwidth}
\footnotesize
\centering
 \begin{tikzpicture}
\directlua{tikzsnake("eneenne",2)}
\end{tikzpicture}
\endminipage\hfill
\minipage[c]{0.45\textwidth}%
\centering
\footnotesize
\begin{tikzpicture}
    \directlua{tikzsnake("eneenne",4)}
\end{tikzpicture}
\endminipage
\end{figure}
    \caption{Example of snake graph and sign function applied to it.}
\end{figure}

\section{Labeled snake graphs from surfaces}
Suppose we now want to construct a snake graph corresponding to an arc on any given triangulated surface. Suppose that $T$ is an ideal triangulation of some surface $(S,M)$; and let $\gamma$ be an arc that is not in $T$, with starting point $s$, and endpoint $t$ (i.e. we are choosing an orientation within our surface), both of which are contained in $M$. 
\begin{figure}[H]
    \centering
\begin{tikzpicture} 
    %\begin{pgfonlayer}{bg}
    \clip (0,0) rectangle (12cm,6cm);
    %\draw[style=help lines] (-14,-14) grid[step=1cm] (14,14);
    %\end{pgfonlayer}

    \draw (1,3) -- (3,5);
    \draw (1,3) -- (3,1);
    \draw (3,5) -- (10,5);
    \draw (3,1) -- (10,1);
    \draw (10,5) -- (12,3);
    \draw (10,1) -- (12,3);

    \node[draw,circle,inner sep=0.5pt,fill, label=left : {\small $s$}] at (1, 3) {};
    \node[draw,circle,inner sep=0.5pt,fill, label=right : {\small $t$}] at (12,3) {};

    \draw[red,thick] (1,3) -- (12,3) node[label= above:{\gamma},pos = 0.5] {};
    \draw (3,1) -- (3,5); 
    \draw (3,1) -- (5,5);
    \draw (5,5) -- (5,1);
    \draw (5,1) -- (7,5);
    \draw (7,5) -- (10,1);
    \draw (7,5) -- (7.5,1);
    \draw (10,1) -- (10,5);

    \node[draw = none,label = $\Delta_0$] at (2.5,3) {};
    \node[draw=none,label = $\Delta_1$] at (3.75,3.5) {};
    \node[draw=none,label = $\Delta_2$] at (4.25,1.5) {};
    \node[draw=none,label = $\Delta_3$] at (5.75,3.5) {};
    \node[draw=none,label = $\Delta_4$] at (6.5,1.5) {};
    \node[draw=none,label = $\Delta_5$] at (8.25,1.5) {};
    \node[draw=none,label = $\Delta_6$] at (9,3.5) {};
    \node[draw=none,label = $\Delta_7$] at (10.5,3.25) {};

    \node[draw = none, label = $e_1$] at (4,5) {};
    \node[draw = none, label = $e_2$] at (4,0) {};
    \node[draw = none, label = $e_3$] at (6,5) {};
    \node[draw = none, label = $e_4$] at (6.25,0) {};
    \node[draw = none, label = $e_5$] at (8.5,0) {};
    \node[draw = none, label = $e_6$] at (8.5,5) {};

    \node[draw = none, label = $\tau_1$] at (2.75,1.75) {};
    \node[draw = none, label = $\tau_2$] at (3.35,2) {};
    \node[draw = none, label = $\tau_3$] at (4.75,2) {};
    \node[draw = none, label = $\tau_4$] at (5.35,2) {};
    \node[draw = none, label = $\tau_5$] at (7,1.75) {};
    \node[draw = none, label = $\tau_6$] at (9.25,1) {};
    \node[draw = none, label = $\tau_7$] at (9.75,2) {};

    \node[draw = none, label = $\color{red}>$] at (6.5,2.59) {};

    
    
\end{tikzpicture}
    
\end{figure}
%\subsection{Weights and expansion formula}
%First we recall a key definition; that of a \emph{matching}. 
%\begin{definition}
    %A \emph{matching} is a collection of edges such that no pair contains a common vertex. A matching is called a \emph{perfect matching} if precisely every vertex is incident with exactly 1 of these edges in the collection. 
%\end{definition}
Let $s = p_0,p_1,\dots,p_{d+1} = t$ be the points, in order, in which $\gamma$ intersects any element of $T$. Then for $i = 1,\dots,d$, let $\tau_i$ be the arc in $T$ containing the point $p_i$; and let $\Delta_{i-1}, \Delta_i$ be the two ideal triangles adjacent to $\tau_i$. Furthermore, for all $i$, the arcs $\tau_i,\tau_{i+1}$ form two sides of the ideal triangle $\Delta_i$; where we denote the third arc by $e_i$. Finally, we let $G_i$ be the quadrilateral in $T$ that contains the arc $\tau_i$ as a diagonal; then $G_i$ is precisely a tile as \ref{tile}. 

We can additionally assign a sign function $f$ on the edges $e_1,\dots,e_d$ by
\begin{equation*}
    f(e_j) = \begin{cases}
        +1 \text{ if $e_j$ lies on the right of $\gamma$ when passing through $\Delta_j$;} \\
        -1 \text{ otherwise.}
    \end{cases}
\end{equation*}
Then, the labeled snake graph $\s_{\gamma} = (G_1,\dots,G_d)$ with sign function $f$ is the \emph{snake graph associated to the arc $\gamma$}. Next, define the \emph{weight} $x(e)$, where $e$ is any edge of $\s_{\gamma}$, to be the cluster variable associated with the arc $\tau(e)$ of $T$. 
\section{Perfect matchings, Height and the Expansion formula}
Recall that a \emph{perfect matching} $P$ of a graph $G$ is a collection of edges in which every vertex of $G$ is incident with exactly $1$ edge in $P$. We denote the set of all perfect matchings of $G$ by $\mathcal{M}(G)$. Moreover, note that every snake graph has precisely 2 perfect matchings that contain boundary edges only. We denote these by $P_-$ and $P_+$\footnote{The choice of which is $P_-$ will only make a difference in the setting of Cluster algebras with non-trivial coefficients.}. Then for a matching $P$ we define $P_-\ominus P = (P_- \cup P)\backslash(P_- \cap P)$ to be the symmetric difference of $P_-$ and $P$. In other words, $P_-\ominus P $ is the set of boundary edges of a subgraph $\s_P$, possibly disconnected and made of tiles $G_i$; i.e. 
\begin{equation*}
    \s_P = \bigcup_{i} G_i.
\end{equation*}
We then define the \emph{height monomial} of $P$ by 
\begin{equation*}
    y(P) = \prod_{G_i \text{ a tile in } \s_P}y_i.
\end{equation*}
Consider an arbitrary marked surface $(S,M)$, with triangulation $T = \{\tau_1,\dots,\tau_n\}$, and we let $\mathcal{A} = \mathcal{A}(\mathbf{x}_T,\mathbf{y}_T,\mathcal{Q}_T)$ be its corresponding Cluster algebra; where $\mathbf{x}_T=(x_1,\dots,x_n)$ correspond to the diagonals in the triangulation $T$, and $\mathbf{y}_T = (y_1,\dots,y_m)$ correspond to the boundary arcs of the marked surface, and are also the basis for the corresponding tropical semifield. In  \cite{MSW}, the authors proved the following;
\begin{theorem}
    Let $\gamma$ be an arc not in the triangulation $T$, such that it intersects the diagonals $\{\tau_{i_1},\dots,\tau_{i_d}\}$. Then the cluster variable $x_{\gamma}$ is equal to 
    \begin{equation}\label{ExpansionFormula}
        x_{\gamma} = \dfrac{1}{\text{cross}(\gamma)} \sum_{P \ \in \ \mathcal{M}(\s_{\gamma})} x(P)y(P).
    \end{equation} 
\end{theorem}
Where $x(P) = \prod_{e \in P}x(e)$ is the weight of $P$ and $y(p)$ is the height of $P$. Moreover, cross$(\gamma)$ is the monomial consisting of the cluster variables that are associated to the arcs $\{\tau_{i_j}\}_{j = 1}^d$; in other words, $\text{cross}(\gamma) = \prod_{j = 1}^dx_{i_j}$. In addition, the authors showed that if we iterate this process over all possible arcs $\gamma$ in the surface $(S,M)$, these $x_{\gamma}$ generate the corresponding Cluster algebra.

\begin{figure}[H]\label{ExpansionExample}
    \centering


\tikzset{every picture/.style={line width=0.75pt}} %set default line width to 0.75pt        

\begin{tikzpicture}[x=0.75pt,y=0.75pt,yscale=-1,xscale=1]
%uncomment if require: \path (0,437); %set diagram left start at 0, and has height of 437

%Straight Lines [id:da4908232703769615] 
\draw    (90.5,210) -- (119.5,250) ;
%Straight Lines [id:da17650145113834725] 
\draw    (90.5,210) -- (121.5,171) ;
%Straight Lines [id:da7174102064637746] 
\draw    (121.5,171) -- (170.5,170) ;
%Straight Lines [id:da8596665591342445] 
\draw    (119.5,250) -- (168.5,249) ;
%Straight Lines [id:da6279303631093601] 
\draw    (199.5,209.95) -- (170.37,170.04) ;
%Straight Lines [id:da7242109050720578] 
\draw    (199.5,209.95) -- (168.63,249.05) ;
%Straight Lines [id:da03863244955219047] 
\draw    (121.5,171) -- (119.5,250) ;
%Straight Lines [id:da31854017369699694] 
\draw    (121.5,171) -- (168.5,249) ;
%Straight Lines [id:da47134283434636637] 
\draw    (121.5,171) -- (199.5,209.95) ;
%Straight Lines [id:da5992871763576858] 
\draw [color={rgb, 255:red, 255; green, 0; blue, 0 }  ,draw opacity=1 ]   (90.5,210) -- (168.71,170.89) ;
\draw [shift={(170.5,170)}, rotate = 153.43] [fill={rgb, 255:red, 255; green, 0; blue, 0 }  ,fill opacity=1 ][line width=0.08]  [draw opacity=0] (12,-3) -- (0,0) -- (12,3) -- cycle    ;
%Shape: Square [id:dp11810124584691684] 
\draw   (246.63,208.3) -- (276.33,208.3) -- (276.33,238) -- (246.63,238) -- cycle ;
%Shape: Square [id:dp13699116965413383] 
\draw   (276.33,208.3) -- (306.03,208.3) -- (306.03,238) -- (276.33,238) -- cycle ;
%Shape: Square [id:dp1790172677132683] 
\draw   (276.33,178.6) -- (306.03,178.6) -- (306.03,208.3) -- (276.33,208.3) -- cycle ;
%Shape: Square [id:dp2883938512500389] 
\draw   (361.88,163.78) -- (377.86,163.78) -- (377.86,179.75) -- (361.88,179.75) -- cycle ;
%Shape: Square [id:dp7088947878360049] 
\draw   (377.86,163.78) -- (393.84,163.78) -- (393.84,179.75) -- (377.86,179.75) -- cycle ;
%Shape: Square [id:dp3954910092087409] 
\draw   (377.86,147.8) -- (393.84,147.8) -- (393.84,163.78) -- (377.86,163.78) -- cycle ;
%Shape: Rectangle [id:dp179337128236425] 
\draw   (427.39,181.35) -- (443.37,181.35) -- (443.37,197.33) -- (427.39,197.33) -- cycle ;
%Shape: Rectangle [id:dp839678338902704] 
\draw   (443.37,181.35) -- (459.34,181.35) -- (459.34,197.33) -- (443.37,197.33) -- cycle ;
%Shape: Square [id:dp03950570157166866] 
\draw   (443.37,165.37) -- (459.34,165.37) -- (459.34,181.35) -- (443.37,181.35) -- cycle ;
%Shape: Square [id:dp6368269407485009] 
\draw   (361.88,221.29) -- (377.86,221.29) -- (377.86,237.27) -- (361.88,237.27) -- cycle ;
%Shape: Square [id:dp5080247361620098] 
\draw   (377.86,221.29) -- (393.84,221.29) -- (393.84,237.27) -- (377.86,237.27) -- cycle ;
%Shape: Square [id:dp3148822004848474] 
\draw   (377.86,205.32) -- (393.84,205.32) -- (393.84,221.29) -- (377.86,221.29) -- cycle ;
%Shape: Rectangle [id:dp447116954358052] 
\draw   (427.39,235.67) -- (443.37,235.67) -- (443.37,251.65) -- (427.39,251.65) -- cycle ;
%Shape: Square [id:dp4061650684840712] 
\draw   (443.37,235.67) -- (459.34,235.67) -- (459.34,251.65) -- (443.37,251.65) -- cycle ;
%Shape: Square [id:dp7947789399156008] 
\draw   (443.37,219.7) -- (459.34,219.7) -- (459.34,235.67) -- (443.37,235.67) -- cycle ;
%Straight Lines [id:da8786639629522955] 
\draw    (403.74,160.9) -- (422.6,179.75) ;
%Straight Lines [id:da09557684843452541] 
\draw    (403.74,212.03) -- (422.6,230.88) ;
%Straight Lines [id:da7301540036749984] 
\draw    (421,189.34) -- (403.42,205.32) ;
%Straight Lines [id:da9619766876540269] 
\draw [color={rgb, 255:red, 255; green, 0; blue, 0 }  ,draw opacity=1 ]   (361.88,163.78) -- (377.86,163.78) ;
%Straight Lines [id:da6633598084478148] 
\draw [color={rgb, 255:red, 255; green, 0; blue, 0 }  ,draw opacity=1 ]   (361.88,179.75) -- (377.86,179.75) ;
%Straight Lines [id:da867888427891212] 
\draw [color={rgb, 255:red, 255; green, 0; blue, 0 }  ,draw opacity=1 ]   (377.86,147.8) -- (393.84,147.8) ;
%Straight Lines [id:da591102205651237] 
\draw [color={rgb, 255:red, 255; green, 0; blue, 0 }  ,draw opacity=1 ]   (443.37,165.37) -- (459.34,165.37) ;
%Straight Lines [id:da6749513848128621] 
\draw [color={rgb, 255:red, 255; green, 0; blue, 0 }  ,draw opacity=1 ]   (377.86,205.32) -- (393.84,205.32) ;
%Straight Lines [id:da7958887267913386] 
\draw [color={rgb, 255:red, 255; green, 0; blue, 0 }  ,draw opacity=1 ]   (377.86,221.29) -- (393.84,221.29) ;
%Straight Lines [id:da4580741088081952] 
\draw [color={rgb, 255:red, 255; green, 0; blue, 0 }  ,draw opacity=1 ]   (377.86,237.27) -- (393.84,237.27) ;
%Straight Lines [id:da08524721193125528] 
\draw [color={rgb, 255:red, 255; green, 0; blue, 0 }  ,draw opacity=1 ]   (443.37,251.65) -- (459.34,251.65) ;
%Straight Lines [id:da7449561570332422] 
\draw [color={rgb, 255:red, 255; green, 0; blue, 0 }  ,draw opacity=1 ]   (393.94,179.65) -- (393.84,163.78) ;
%Straight Lines [id:da7201215013298098] 
\draw [color={rgb, 255:red, 255; green, 0; blue, 0 }  ,draw opacity=1 ]   (361.99,237.16) -- (361.88,221.29) ;
%Straight Lines [id:da9147778038385569] 
\draw [color={rgb, 255:red, 255; green, 0; blue, 0 }  ,draw opacity=1 ]   (427.39,197.33) -- (427.28,181.46) ;
%Straight Lines [id:da1736214385948459] 
\draw [color={rgb, 255:red, 255; green, 0; blue, 0 }  ,draw opacity=1 ]   (443.37,197.33) -- (443.26,181.46) ;
%Straight Lines [id:da7301284550283822] 
\draw [color={rgb, 255:red, 255; green, 0; blue, 0 }  ,draw opacity=1 ]   (459.34,197.33) -- (459.24,181.46) ;
%Straight Lines [id:da8545909194796913] 
\draw [color={rgb, 255:red, 255; green, 0; blue, 0 }  ,draw opacity=1 ]   (443.37,235.67) -- (443.26,219.8) ;
%Straight Lines [id:da4475029716801513] 
\draw [color={rgb, 255:red, 255; green, 0; blue, 0 }  ,draw opacity=1 ]   (459.45,235.57) -- (459.34,219.7) ;
%Straight Lines [id:da433658880536243] 
\draw [color={rgb, 255:red, 255; green, 0; blue, 0 }  ,draw opacity=1 ]   (427.5,251.54) -- (427.39,235.67) ;


% Text Node
\draw (140,186.4) node [anchor=north west][inner sep=0.75pt]    {$\textcolor[rgb]{0.96,0,0}{\gamma }$};
% Text Node
\draw (90,224.4) node [anchor=north west][inner sep=0.75pt]    {$a$};
% Text Node
\draw (96,169.4) node [anchor=north west][inner sep=0.75pt]    {$b$};
% Text Node
\draw (142,151.4) node [anchor=north west][inner sep=0.75pt]    {$c$};
% Text Node
\draw (191,173.4) node [anchor=north west][inner sep=0.75pt]    {$d$};
% Text Node
\draw (186,228.4) node [anchor=north west][inner sep=0.75pt]    {$e$};
% Text Node
\draw (136,253.4) node [anchor=north west][inner sep=0.75pt]    {$f$};
% Text Node
\draw (122,223.4) node [anchor=north west][inner sep=0.75pt]    {$1$};
% Text Node
\draw (161,216.4) node [anchor=north west][inner sep=0.75pt]    {$2$};
% Text Node
\draw (175,201.4) node [anchor=north west][inner sep=0.75pt]    {$3$};
% Text Node
\draw (256.13,214.38) node [anchor=north west][inner sep=0.75pt]    {$1$};
% Text Node
\draw (285.88,214.38) node [anchor=north west][inner sep=0.75pt]    {$2$};
% Text Node
\draw (285.88,185.38) node [anchor=north west][inner sep=0.75pt]    {$3$};
% Text Node
\draw (236.88,214.38) node [anchor=north west][inner sep=0.75pt]    {$b$};
% Text Node
\draw (255.88,239.38) node [anchor=north west][inner sep=0.75pt]    {$a$};
% Text Node
\draw (255.88,192.38) node [anchor=north west][inner sep=0.75pt]    {$2$};
% Text Node
\draw (265.88,184.38) node [anchor=north west][inner sep=0.75pt]    {$2$};
% Text Node
\draw (270.88,212.38) node [anchor=north west][inner sep=0.75pt]    {$f$};
% Text Node
\draw (284.88,237.38) node [anchor=north west][inner sep=0.75pt]    {$1$};
% Text Node
\draw (286.88,202.38) node [anchor=north west][inner sep=0.75pt]    {$e$};
% Text Node
\draw (306.03,214.7) node [anchor=north west][inner sep=0.75pt]    {$3$};
% Text Node
\draw (306.03,184) node [anchor=north west][inner sep=0.75pt]    {$d$};
% Text Node
\draw (286.88,163.05) node [anchor=north west][inner sep=0.75pt]    {$c$};
% Text Node
\draw (343.75,151.65) node [anchor=north west][inner sep=0.75pt]  [font=\footnotesize]  {$x_{2} x_{3}$};
% Text Node
\draw (431.75,151.65) node [anchor=north west][inner sep=0.75pt]  [font=\footnotesize]  {$x_{3} y_{1}$};
% Text Node
\draw (355.99,239.56) node [anchor=north west][inner sep=0.75pt]  [font=\footnotesize]  {$x_{1} y_{1} y_{2}$};
% Text Node
\draw (409.99,254.56) node [anchor=north west][inner sep=0.75pt]  [font=\footnotesize]  {$x_{1} x_{2} y_{1} y_{2} y_{3}$};
% Text Node
\draw (410.65,162.45) node [anchor=north west][inner sep=0.75pt]  [font=\tiny,color={rgb, 255:red, 108; green, 98; blue, 98 }  ,opacity=0.58 ]  {$y_{1}$};
% Text Node
\draw (398.65,189.45) node [anchor=north west][inner sep=0.75pt]  [font=\tiny,color={rgb, 255:red, 108; green, 98; blue, 98 }  ,opacity=0.58 ]  {$y_{2}$};
% Text Node
\draw (412.65,214.45) node [anchor=north west][inner sep=0.75pt]  [font=\tiny,color={rgb, 255:red, 108; green, 98; blue, 98 }  ,opacity=0.58 ]  {$y_{3}$};


\end{tikzpicture}

\caption{In this figure we have a triangulated hexagon and an arc $\gamma$ (left), with its corresponding labeled snake graph $\s_{\gamma}$(center); and its perfect matchings with their corresponding monomials calculated through the expansion formula \ref{ExpansionFormula}.}
\end{figure}

To better explain how the height monomial works, we can see in the above figure that from the top left perfect matching to the top right one, we are rotating tile 1, which gives the $y_1$ term. To then go to the bottom left, we rotate tile 2, which yields the term $y_2$. Finally, to go from bottom left to bottom right, we rotate tile 3 which yields the $y_3$ term. Moreover, the weight is calculated by looking at which edges are contained in the perfect matching. For example, in the top left perfect matching, we see that it contains edge 2 and 3 of our triangulation; these yield the $x_2$ and $x_3$ terms.

Finally, from Figure 2.7, we obtain that
\begin{equation*}
    x_{\gamma} = \dfrac{x_2x_3 + x_3y_1 + x_1y_1y_2 + x_1x_2y_1y_2y_3}{x_1x_2x_3}.
\end{equation*}

In the case of triangulated polygons, it is clear that there are finitely many arcs, up to isotopy; whilst in the case of, for example, an annulus, there are infinitely many arcs; e.g. we can have a curve that loops around the inner boundary component $n$ times. This is precisely why their corresponding Cluster algebras are finite and infinite, respectively. 

\section{Skein relations}
Suppose we have two arcs $\gamma_1,\gamma_2$ such that they intersect at some point $x$. Then we can define a concept called \emph{smoothing at $x$}; which is a collection of pairs $\{\gamma_3,\gamma_4\}$ and $\{\gamma_5,\gamma_6\}$ obtained from $\{\gamma_1,\gamma_2\}$ by replacing the crossing \KPB; in a small neighborhood of $x$, with the pair \KPC and \KPD. 
\begin{figure}[H]
    \centering


\tikzset{every picture/.style={line width=0.75pt}} %set default line width to 0.75pt        

\begin{tikzpicture}[x=0.75pt,y=0.75pt,yscale=-1,xscale=1]
%uncomment if require: \path (0,300); %set diagram left start at 0, and has height of 300

%Shape: Circle [id:dp7164727092292058] 
\draw  [fill={rgb, 255:red, 228; green, 227; blue, 188 }  ,fill opacity=1 ] (63,133.8) .. controls (63,126.23) and (69.13,120.1) .. (76.7,120.1) .. controls (84.27,120.1) and (90.4,126.23) .. (90.4,133.8) .. controls (90.4,141.37) and (84.27,147.5) .. (76.7,147.5) .. controls (69.13,147.5) and (63,141.37) .. (63,133.8) -- cycle ;
%Curve Lines [id:da4188364150873798] 
\draw    (29.9,136.15) .. controls (38.4,146.65) and (170.9,185.15) .. (61.4,94.15) ;
%Curve Lines [id:da7397926385822463] 
\draw    (81.4,172.65) .. controls (35.4,117.15) and (42.2,92.35) .. (122.9,133.15) ;
%Straight Lines [id:da5533429251019936] 
\draw    (174.4,91.15) -- (250.8,140.35) ;
%Curve Lines [id:da9637986309064092] 
\draw    (146.5,147.65) .. controls (369.1,175.8) and (62.8,28.85) .. (218,186.65) ;
%Curve Lines [id:da5730466699163983] 
\draw    (307.1,98.95) .. controls (352.52,118.28) and (351.3,147.6) .. (351.3,178.6) ;
%Curve Lines [id:da377050407824876] 
\draw    (282.1,142.2) .. controls (322.1,166.2) and (336.3,160.54) .. (378.1,136.2) ;
%Straight Lines [id:da4567118907128783] 
\draw    (172.4,178.15) -- (248.8,227.35) ;
%Straight Lines [id:da9944916425980186] 
\draw    (155.4,227.15) -- (231.8,276.35) ;
%Shape: Circle [id:dp4153786288701671] 
\draw   (176.2,224.68) .. controls (176.2,212.46) and (186.11,202.55) .. (198.32,202.55) .. controls (210.54,202.55) and (220.45,212.46) .. (220.45,224.68) .. controls (220.45,236.89) and (210.54,246.8) .. (198.32,246.8) .. controls (186.11,246.8) and (176.2,236.89) .. (176.2,224.68) -- cycle ;
%Curve Lines [id:da007991355624611951] 
\draw    (278.7,230.6) .. controls (332.7,164.3) and (360.7,230.3) .. (323.7,262.6) ;
%Straight Lines [id:da05879559895469377] 
\draw    (291.4,175.15) -- (367.8,224.35) ;
%Curve Lines [id:da694044951364646] 
\draw    (437.24,184.34) .. controls (489.7,249.3) and (434.7,257.3) .. (400.72,225.76) ;
%Straight Lines [id:da6309735851468482] 
\draw    (491.06,202.76) -- (434.16,273.61) ;
%Straight Lines [id:da32536832233258217] 
\draw    (515.1,220) -- (593.1,272) ;
%Straight Lines [id:da04214537825490794] 
\draw    (540.4,185.15) -- (616.8,234.35) ;
%Shape: Circle [id:dp055172114017479124] 
\draw  [fill={rgb, 255:red, 228; green, 227; blue, 188 }  ,fill opacity=1 ] (184,133.8) .. controls (184,126.23) and (190.13,120.1) .. (197.7,120.1) .. controls (205.27,120.1) and (211.4,126.23) .. (211.4,133.8) .. controls (211.4,141.37) and (205.27,147.5) .. (197.7,147.5) .. controls (190.13,147.5) and (184,141.37) .. (184,133.8) -- cycle ;
%Shape: Circle [id:dp06546394326732974] 
\draw  [fill={rgb, 255:red, 228; green, 227; blue, 188 }  ,fill opacity=1 ] (304,133.8) .. controls (304,126.23) and (310.13,120.1) .. (317.7,120.1) .. controls (325.27,120.1) and (331.4,126.23) .. (331.4,133.8) .. controls (331.4,141.37) and (325.27,147.5) .. (317.7,147.5) .. controls (310.13,147.5) and (304,141.37) .. (304,133.8) -- cycle ;
%Shape: Circle [id:dp10396248031580357] 
\draw  [fill={rgb, 255:red, 228; green, 227; blue, 188 }  ,fill opacity=1 ] (184.63,224.68) .. controls (184.63,217.11) and (190.76,210.98) .. (198.32,210.98) .. controls (205.89,210.98) and (212.02,217.11) .. (212.02,224.68) .. controls (212.02,232.24) and (205.89,238.37) .. (198.32,238.37) .. controls (190.76,238.37) and (184.63,232.24) .. (184.63,224.68) -- cycle ;
%Shape: Circle [id:dp10282109383153959] 
\draw  [fill={rgb, 255:red, 228; green, 227; blue, 188 }  ,fill opacity=1 ] (304,224.8) .. controls (304,217.23) and (310.13,211.1) .. (317.7,211.1) .. controls (325.27,211.1) and (331.4,217.23) .. (331.4,224.8) .. controls (331.4,232.37) and (325.27,238.5) .. (317.7,238.5) .. controls (310.13,238.5) and (304,232.37) .. (304,224.8) -- cycle ;
%Shape: Circle [id:dp6810690428961315] 
\draw  [fill={rgb, 255:red, 228; green, 227; blue, 188 }  ,fill opacity=1 ] (424,223.8) .. controls (424,216.23) and (430.13,210.1) .. (437.7,210.1) .. controls (445.27,210.1) and (451.4,216.23) .. (451.4,223.8) .. controls (451.4,231.37) and (445.27,237.5) .. (437.7,237.5) .. controls (430.13,237.5) and (424,231.37) .. (424,223.8) -- cycle ;
%Shape: Circle [id:dp8686972692006888] 
\draw  [fill={rgb, 255:red, 228; green, 227; blue, 188 }  ,fill opacity=1 ] (545,223.8) .. controls (545,216.23) and (551.13,210.1) .. (558.7,210.1) .. controls (566.27,210.1) and (572.4,216.23) .. (572.4,223.8) .. controls (572.4,231.37) and (566.27,237.5) .. (558.7,237.5) .. controls (551.13,237.5) and (545,231.37) .. (545,223.8) -- cycle ;

% Text Node
\draw (133.4,123.75) node [anchor=north west][inner sep=0.75pt]    {$=$};
% Text Node
\draw (266.2,124.4) node [anchor=north west][inner sep=0.75pt]    {$+$};
% Text Node
\draw (134,215.4) node [anchor=north west][inner sep=0.75pt]    {$=$};
% Text Node
\draw (268.2,233.4) node [anchor=north west][inner sep=0.75pt]    {$+$};
% Text Node
\draw (378.2,229.4) node [anchor=north west][inner sep=0.75pt]    {$+$};
% Text Node
\draw (499.2,226.4) node [anchor=north west][inner sep=0.75pt]    {$+$};


\end{tikzpicture}


\caption{Example of a smoothing process.}
\end{figure}
In \cite{MW}, the authors proved the following; also known as the \emph{Skein relations};
\begin{theorem}[Skein relations]\label{Skeinrelations}
    If $\{\gamma_3,\gamma_4\}$ and $\{\gamma_5,\gamma_6\}$ are curves obtained from $\{\gamma_1,\gamma_2\}$ through a smoothing process, then 
    \begin{equation*}
        x_{\gamma_1}x_{\gamma_2} = y_{34}x_{\gamma_3}x_{\gamma_4} + y_{56}x_{\gamma_5}x_{\gamma_6};
    \end{equation*}
    for some $y_{34},y_{56}$ in $\mathcal{G}$ and $x_{\gamma_i}$ is the cluster algebra element corresponding to the arc $\gamma_i$. Moreover, we  if we apply this to a self-crossing curve $\gamma$, through which we obtain $\gamma_1$ and $\gamma_2$, then
    \begin{equation*}
        x_{\gamma} = y_1x_{\gamma_1} + y_2x_{\gamma_2};
    \end{equation*}
    for some $y_1,y_2$ in $\mathcal{G}$. 
\end{theorem}
It turns out that, by using Skein relations, we find that a product of cluster variables corresponds to an arc in the surface. If we let $\mathcal{B}$ be the collection of all the curves that arise by applying the smoothing process to all possible curves, we get that $\mathcal{B}$ forms a basis for the cluster algebra $\mathcal{A} = \mathcal{A}(\mathbf{x},\mathbf{y},\mathcal{Q})$. This has been proven in \cite{MSW}.
\chapter{Continued fractions}
A \emph{continued fraction} is a particularly useful tool in number theory; it allows us to represent any real number as a sequence; more precisely, 
\begin{equation*}
    [a_1,a_2,\dots,a_n] = a_1 + \dfrac{1}{a_2+\dfrac{1}{\ddots + \dfrac{1}{a_n}}}.
\end{equation*}
For example, consider the real numbers $22/7,\sqrt{2}$; then, $22/7 = 3+1/7 = [3;7]$, and 
\begin{align*}
    \sqrt{2} &= 1 + \dfrac{1}{2+ \dfrac{1}{2 + \ddots}} = [1;2,2,2,\dots].
\end{align*}
The reader may notice that for $\alpha \in \mathbb{R}$, then the continued fraction of $\alpha$ is finite if and only if $\alpha \in \mathbb{Q}$. In fact, if $\alpha = p/q$, for $p,q \in \mathbb{Z}$, then the continued fraction algorithm is nothing but the Euclidean algorithm applied to $p$ and $q$. 

A continued fraction $[a_1,\dots,a_n]$ is \emph{positive} if $a_i \in \mathbb{Z}_{\geq 0}$ for all $i$; and we say it is \emph{simple} if $a_1 \in \mathbb{Z}$ and $a_j \in \mathbb{Z}_{\ge 1}$ for $2 \leq j \leq n$. Moreover, if $a_n = 1$, and clearly $1/1 = 1$, it holds that $[a_1,\dots,a_{n-1},1] = [a_1,\dots,a_{n-1}+1]$. This identity yields the following classical result;
\begin{theorem}[\cite{HW}, Theorem 162 p.g. 136]\label{thm3.4}~ 
    \begin{itemize}
        \item[(i)] There exists a bijection between the set of rational numbers greater than 1 $\mathbb{Q}_{> 1}$, and the set of positive, finite continued fractions whose last coefficients (e.g. $a_n$ in the paragraph above) is at least 2. 
        \item[(ii)] There exists a bijection between the set of rational numbers $\mathbb{Q}$, and the set of simple, finite continued fractions whose last coefficient is at least 2. 
    \end{itemize}
\end{theorem}
\section{Snake graphs of a continued fraction}
In \cite{CS2} the construction of a snake graph, by using the sign function described in the previous chapter, corresponding to a continued fraction is described; it is done in a way such that the number of perfect matchings is equal to the numerator of the fraction. For a continued fraction $[a_1,\dots,a_n]$, we denote by $\mathcal{S}[a_1,\dots,a_n]$ its corresponding snake graph. 

Consider $[a_1,\dots,a_n]$, and the sequence 
\begin{equation}\label{signsequence}
\begin{array}{cccccccc}
 %(f(e_0),f(e_1),\ldots,f(e_d)) &=&
  ( \underbrace{ -,\ldots,-},&  \underbrace{ + ,\dots, +},&  \underbrace{ -,\ldots,-},& \ldots,&  \underbrace{\epsilon,\ldots,\epsilon}) ,  \\
 a_1 & a_2 & a_3&\ldots&a_n
\end{array} 
\end{equation}
where $\epsilon = $
$
\begin{cases}
+  \ \text{if} \ n \ \text{is even}; \\
- \ \text{if} \ n \ \text{is odd}
\end{cases}.
$
\\

Then, the snake graph $\mathcal{S}[a_1,\dots,a_n]$ is the graph with precisely $a_1 + \dots + a_n - 1$ tiles determined by the sign sequence \ref{signsequence}. For the reader's understanding, we provide a worked out example. 

\begin{example}\label{Ex3.5}
Consider the fraction $31/7$, with its corresponding continued fraction $[4,2,3]$. We get the sign sequence
$
\begin{array}{ccc}
  ( -,-,-,-,&  +,+,&  -,-,-);
\end{array}
$
which yields the following snake graph (on the left);
\begin{figure}[H]
    \centering
    \begin{figure}[!htb]
\centering
\minipage[c]{0.5\textwidth}
\footnotesize
\centering
 \begin{tikzpicture}
\directlua{tikzsnake("eneenne",4)}
\end{tikzpicture}
\endminipage\hfill
\minipage[c]{0.5\textwidth}%
\centering
\footnotesize
\begin{tikzpicture}
    \directlua{tikzsnake("eneenne",5)}
    \node at (B0) {$2$};
	\node at (B1) {$3$};
	\node at (B2) {$4$};
	\node at (B3) {$5$};
	\node at (B4) {$9$};
	\node at (B5) {$13$};
	\node at (B6) {$22$};
	\node at (B7) {$31$};
\end{tikzpicture}
\endminipage
\end{figure}
\end{figure}
\end{example}
On the right, we have the snake graph in which the number at tile $G_i$ indicates the number of perfect matchings of the subsnake graph given by the first $i$ tiles.  
We can take the above result to yield and even stronger condition on the relation between continued fractions and snake graphs. As mentioned above, the number of perfect matchings of a snake graph $\mathcal{S}[a_1,\dots,a_n]$ is precisely equal to the numerator of the continued fraction $[a_1,\dots, a_n]$. Furthermore, in \cite{CS2}, the authors prove the following result;
\begin{theorem}\label{thm3.6}
    If $m(\mathcal{S})$ denotes the number of perfect matchings of $\mathcal{S}$, then
    \begin{equation*}
        [a_1,\dots,a_n] = \dfrac{m(\mathcal{S}[a_1,\dots,a_n])}{m(\mathcal{S}[a_2,\dots,a_n])}.
    \end{equation*}
\end{theorem}
In other words, the number of perfect matchings of the snake graph $\mathcal{S}[a_2,\dots,a_n]$ is equal to the denominator of the continued fraction $[a_1,a_2,\dots,a_n]$. Applying it to Example \ref{Ex3.5}, we get the continued fraction $[2,3]$, with sign sequence $
\begin{array}{cc}
  (+,+,&  -,-,-);
\end{array}
$; which yields the following
\begin{figure}[H]
    \centering
    \begin{figure}[!htb]
\centering
\minipage[c]{0.5\textwidth}
\footnotesize
\centering
 \begin{tikzpicture}
\directlua{tikzsnake("een",4)}
\end{tikzpicture}
\endminipage\hfill
\minipage[c]{0.5\textwidth}%
\centering
\footnotesize
\begin{tikzpicture}
    \directlua{tikzsnake("een",5)}
    \node at (B0) {$2$};
	\node at (B1) {$3$};
	\node at (B2) {$5$};
	\node at (B3) {$7$};
\end{tikzpicture}
\endminipage
\end{figure}
\end{figure}
as required. 
Finally, let $f$ be the map from a snake graph $\s$ to a continued fraction by the sign sequence; then Theorem \ref{thm3.4} can be represented and strengthened via the following result; 
\begin{theorem}[\cite{CS2}, Theorem 4.1]\label{thm3.7}
    There is a commutative diagram;
\[\begin{tikzcd}
	\left\{\begin{array}{c}
\text{pairs $(\s,e_d)$ of a snake graph $\s$} \\ \text{with $d \geq 1$ tiles, and $e_d$ in last tile}
\end{array}\right\} &&& \left\{\begin{array}{c}
\text{positive continued fractions different} \\ \text{from the continued fraction $[1]$}
\end{array}\right\} \\
	\\
	\left\{\begin{array}{c}
\text{snake graphs $\s$ with} \\ \text{at least $1$ tile}
\end{array}\right\} &&& \left\{\begin{array}{c}
\text{positive continued fractions} \\ \text{with last coefficient $>1$}
\end{array}\right\} \\
	&&& \mathbb{Q}_{>1}
	\arrow["\chi"', from=3-1, to=4-4]
	\arrow["{\text{forget} \ e_d}"', two heads, from=1-1, to=3-1]
	\arrow["g", two heads, from=1-4, to=3-4]
	\arrow["{F'}", from=3-1, to=3-4]
	\arrow["Ev"', from=3-4, to=4-4]
	\arrow["G"', shift right=3, from=1-1, to=1-4]
	\arrow["F", shift left=2, from=1-1, to=1-4]
	\arrow["\cong"{pos=0.6}, from=3-1, to=4-4]
	\arrow["\cong"', shift left=3, from=1-4, to=1-1]
	\arrow["\cong", from=3-4, to=4-4]
\end{tikzcd}\]
where the maps are defined as follows:
\begin{itemize}
    \item $F$ maps the pair $(\s,e_d)$ to the continued fraction defined by the sign sequence 
    \begin{equation*}
        (f(e_0), f(e_1), \dots, f(e_d));
    \end{equation*} 
    \item $F'$ maps the snake graph $\s$ to the continued fraction defined by the sign sequence 
    \begin{equation*}
        (f(e_0),\dots,f(e_{d-1}),f(e_{d-1})).
    \end{equation*} 
    \item $G$ sends $[a_1,\dots,a_n]$ to the pair consisting of the snake graph $\s[a_1,\dots,a_n]$ and an edge $e_d$ determined by the sign sequence. 
    \item $g$ is defined by
    \begin{equation*}
        g([a_1,\dots,a_n]) = \begin{cases}
            [a_1,\dots,a_{n-1} + 1], \ &\text{if } a_n = 1 \\
            [a_1,\dots,a_n] \ &\text{if } a_n > 1.
        \end{cases}
    \end{equation*} 
    \item $\chi$ maps a snake graph $\s$ to the quotient 
    \begin{equation*}
        \dfrac{m(\s)}{m\left(\s \backslash \left\{\begin{array}{c}
\text{first zigzag} \\ \text{subsnake graph}
\end{array}\right\}\right)}
    \end{equation*}
    \item $Ev$ is the bijection in Theorem \ref{thm3.4}; which sends a continued fraction to its value.
\end{itemize}
Additionally, $F,G,F',\chi$ and $Ev$ are bijections.
\end{theorem}
To better understand how the map $\chi$ works, recall Example \ref{Ex3.5}, i.e. we have the fraction $31/7 = [4,2,3]$, and notice that its first zigzag subsnake graph is precisely that determined by the first 4 tiles; so that we obtain 
\begin{equation*}
  \s \backslash \left\{\begin{array}{c}
\text{first zigzag} \\ \text{subsnake graph}
\end{array}\right\} = 
  \begin{tikzpicture}[baseline={([yshift=-.5ex]current bounding box.center)},vertex/.style={anchor=base,
    circle,fill=black!25,minimum size=18pt,inner sep=2pt}]
    \directlua{tikzsnake("nne",5)}
  \end{tikzpicture}
\end{equation*}
which corresponds to the continued fraction $[1,1,3]$; hence we have precisely that $\chi([4,2,3]) = m(\s[4,2,3])/m(\s[1,1,3]) = 31/7$. This is quite intuitive as a zigzag subsnake graph has a sign sequence of the form $(\pm,\dots,\pm)$; in other words, the corresponding continued fraction is of length 1; thus $\chi$ is essentially removing the first entry of a continued fraction, almost applying Theorem \ref{thm3.6}.
\section{Palindromification}
Observe that given any snake graph $\mathcal{S}$, rotating it by $180^{\circ}$ yields an isomorphic snake graph. Similarly, if we mirror $\mathcal{S}$ over the lines $y = x$, and $y = -x$, we also obtain isomorphic snake graph. Moreover, observe that;
\begin{align*}
    [a_1,\dots,a_n] &= a_1 + \dfrac{1}{a_2+\dfrac{1}{\dots + \dfrac{1}{a_n}}}\\
    &= a_1 + \dfrac{1}{a_2+\dfrac{1}{\dots + \dfrac{1}{a_n-1 + \dfrac{1}{1} }}}= [a_1,\dots,a_n -1 , 1].
\end{align*}
This yields to the following observation;
\begin{theorem}\label{thm3.8}
    We have the following isomorphisms; where $e_d$ is the edge of the final tile:
\begin{itemize}
    \item[(a)] Mirror over $y=x$;
    \begin{equation*}
        \mathcal{S}[a_1,\dots,a_n] \cong \mathcal{S}[1,a_1-1,a_2,\dots,a_n].
    \end{equation*}
    \item[(b)] Mirror over $y = -x$;
    \begin{equation*}
        \mathcal{S}[a_1,\dots,a_n] \cong 
        \begin{cases}
            \mathcal{S}[1,a_n-1,\dots,a_2,a_1] \ &\text{if} \ e_d \ \text{is north}; \\
            \mathcal{S}[a_n,\dots,a_1] \ &\text{if} \ e_d \ \text{is east.}
        \end{cases}
    \end{equation*}
    \item[(c)] Rotation by $180^{\circ}$;
    \begin{equation*}
        \mathcal{S}[a_1,\dots,a_n] \cong 
        \begin{cases}
            \mathcal{S}[1,a_n-1,\dots,a_2,a_1] \ &\text{if} \ e_d \ \text{is east}; \\
            \mathcal{S}[a_n,\dots,a_1] \ &\text{if} \ e_d \ \text{is north.}
        \end{cases}
    \end{equation*}
\end{itemize}
\end{theorem}
\begin{proof}
\begin{itemize}
    \item[(a)] Consider $\mathcal{S} = \mathcal{S}[a_1,\dots,a_n]$; then its corresponding sign sequence is 
    \begin{equation*}
        \begin{array}{cccccccc}
  ( \underbrace{ -,\ldots,-},&  \underbrace{ + ,\dots, +},&  \underbrace{ -,\ldots,-},& \ldots,&  \underbrace{\pm,\ldots,\pm}) ;  \\
 a_1 & a_2 & a_3&\ldots&a_n
\end{array}
    \end{equation*}
    if we mirror $\mathcal{S}$ over the line $y = x$, we notice that we obtain the sign sequence
    \begin{equation*}
        \begin{array}{cccccccc}
  (  -,&  \underbrace{ + ,\dots, +},&  \underbrace{ -,\ldots,-},& \ldots,&  \underbrace{\mp,\ldots,\mp}) . \\
 &a_1 -1 & a_2 & \ldots&a_n
\end{array}
    \end{equation*}
    Therefore, for $a_1 > 1$, it holds that this is an isomorphism of snake graphs. For the case when $a_1 = 1$, notice that if this processed is reversed, it yields an isomorphism.
    \item[(b)] Define $\Tilde{\mathcal{S}}$ to be the snake graph $\mathcal{S}$ after being mirrored over the line $y = -x$. Let $\Tilde{e}_1,\dots,\Tilde{e}_{d-1}$ be the inner edges of $\Tilde{\mathcal{S}}$; and let $e_0$ be the south edge of the first tile, $G_1$, of $\s$. By mirroring, we obtain a map $\s \xrightarrow{\varphi} \Tilde{\s}$, such that it maps the first tile of $\s$, to the last tile of $\Tilde{\s}$. Say $e_d$ is the east edge of the last tile, of $\s$; then under $\varphi$, it is mapped to the south edge of the first tile of $\Tilde{\s}$; say $\Tilde{e}_0$. Conversely, if $e_d$ is the north edge of the last tile of $\s$, then it is mapped to the west edge of the first tile of $\Tilde{\s}$. In either cases, we have $\Tilde{\s} = \s[a_n,\dots,a_1]$.
    \item[(c)] This follows from a similar reasoning to (b).
\end{itemize}
\end{proof}
Consequently, since we have that $\s[a_1,\dots,a_n] \cong \s[a_n,\dots,a_1]$, via one of the appropriate isomorphisms above, then we can conclude that $m(\s[a_1,\dots,a_n]) = m(\s[a_n,\dots,a_1])$; which by Theorem \ref{thm3.6}, implies that the continued fractions $[a_1,\dots,a_n]$ and $[a_n,\dots,a_1]$ have the same numerator. This yields the following corollay;
\begin{corollary}
    The continued fractions $[a_1,\dots,a_n]$ and $[a_n,\dots,a_1]$ have the same numerator.
\end{corollary}
Now consider $[a_1.\dots,a_n]$; if $n$ is even, then the continued fraction is said to be of \emph{even length}; moreover, it is \emph{palindromic} if $(a_1,\dots,a_n) = (a_n , \dots, a_1)$. Its corresponding snake graph $\s=\s[a_1,\dots,a_n]$ is then called \emph{palindromic of even length}. Lastly, we say that $\s$ has a rotational symmetry at its center tile if $\s$ has a tile $G_i$ such that rotation by $180^{\circ}$ is an automorphism. Note that the number of tiles must be odd in order to have a center tile; i.e., if $d$ is the total number of tiles then  if $G_i$ is the center tile we must have that $i = (d+1)/2$.

\begin{figure}[!htb]
   \begin{minipage}{0.48\textwidth}
     \centering
 \begin{tikzpicture}    
    \directlua{tikzsnake("enne",5)}
    \end{tikzpicture}
   \end{minipage}\hfill
   \begin{minipage}{0.48\textwidth}
     \centering
     \begin{tikzpicture}    
    \directlua{tikzsnake("eeenneee",5)}
    \end{tikzpicture}
   \end{minipage}
   \caption{\label{fig3.1}Examples of snake graphs that have a rotational symmetry at their center tile.}
\end{figure}
   \begin{theorem}\label{thm3.10}
       A snake graph $\s$ is palindromic of even length if and only if $\s$ has a rotational symmetry at its center tile.
   \end{theorem}
   \begin{proof}
       First, suppose $\s = \s[a_1,\dots,a_n,a_n,\dots,a_1]$ is a palindromic snake graph of even length. Let $d$ be the number its number of tiles and observe that by definition, $d = a_1+\dots+a_n + a_n + \dots+ a_1 -1 = 2(a_1,\dots,a_n) -1$; so we have that $d$ is odd; then let $G_i$ be its center tile and notice that $i = (d+1)/2$. Observe that the subsnake graph consisting of the first $i-1$ tiles is isomorphic to $\s[a_1,\dots,a_n]$; and similarly the subsnake graph consisting of the last $i-1$ tiles is then isomorphic to $\s[a_n,\dots,a_1]$. Consequently, note that the subsnake graph formed by the tiles $G_{i-1},G_i,G_{i+1}$ is isomorphic to $\s[2,2]$; so the interior edges $e_{i-1}$ and $e_i$ are parallel; and since $e_{i-1}$ is the last interior edge of $\s[a_1,\dots,a_n]$, and $e_i$ is the first interior edge of $\s[a_n,\dots,a_1]$, it holds that $e_0$ and $e_d$ are parallel. Recall that $e_0$ is the south exterior edge of the first tile $G_1$, so $e_d$ must be the north edge of the last tile $G_d$. By Theorem \ref{thm3.8}, we have that rotation by $180^{\circ}$ at tile $G_i$ is an automorphism.
       \\

       On the other hand, suppose $\s$ has a rotational symmetry at its center tile $G_i$; then it is clear that the tiles $G_{i-1},G_i,G_{i+1}$ form a snake graph that is isomorphic to $\s[2,2]$; so the interior edges $e_{i-1}$ and $e_i$ have different signs. Define $\s[a_1,\dots,a_j]$ to be the snake graph consisting of the first $i-1$ tiles; and $\s[a_{j+1},\dots,a_n]$ that formed by the last $i-1$ tile. Then we must have that $\s$ is of the form $\s[a_1,\dots,a_j,a_{j+1},\dots,a_n]$. By rotational symmetry, we have $(a_1,\dots,a_j) = (a_n,\dots,a_{j+1})$; as required. 
   \end{proof}
To illustrate \ref{thm3.10}, notice that in figure \ref{fig3.1}, on the left, we have the snake graph $\s[3,3]$, and on the right we have $\s[2,1,2,2,1,2]$; both of which are palindromic of even length. Next, consider a snake graph $\s=\s[a_1,\dots,a_n]$. We define the \emph{palindromification} of $\s$, $\s_{\leftrightarrow}$, to be $\s_{\leftrightarrow} = \s[a_n,\dots,a_1,a_1,\dots,a_n]$; that is, we glue two copies of $\s$ together, via a new center tile. 
\\

Let $b_i$ be the single edge corresponding to the tile $G_{l_i}$ in $\s[a_1,\dots,a_n]$, $b_0$ the unique edge in the first tile $G_1$ apart from the edge $e_0$, and $b_n$ the unique edge in the last tile $G_d$ apart from the edge $e_d$. In \cite{CS2} the authors proved the following identity, through a process called \emph{grafting}(see also \cite{CS3});
\begin{theorem}\label{thm3.11}
    If we set $b_0 = \s[a_1,\dots,a_0]$, and $b_n = \s[a_{n+1},\dots,a_n]$, we obtain the following identity;
\begin{equation*}
    b_i\s[a_1,\dots,a_n] = \s[a_1,\dots,a_i]\s[a_{i+1},\dots,a_n]+\s[a_1,\dots,a_{i-1}]\s[a_{i+2},\dots,a_n].
\end{equation*}
\end{theorem}
Through the above theorem, notice that if we apply it to $\s_{\leftrightarrow} = \s[a_n,\dots,a_1,a_1,\dots,a_n]$ for $i=n$, we get the following;
\begin{equation*}
    b_n\s[a_n,\dots,a_1,a_1,\dots,a_n] = \s[a_n,\dots,a_n]\s[a_1,\dots,a_n]+\s[a_n,\dots,a_2]\s[a_2,\dots,a_n].
\end{equation*}
that is, by symmetry, we have 
\begin{equation}
    \s[a_1,\dots,a_n]^2 + \s[a_2,\dots,a_n]^2 = \s\s + \Tilde{\s}\Tilde{\s}
\end{equation}
so we obtain that $m(\s_{\leftrightarrow}) = m(\s)^2 + m(\Tilde{\s})^2$; where $\Tilde{\s} = \s[a_2,\dots,a_n]$. This leads to the following result;
\begin{theorem}\label{thm3.12}
    Let $\s = \s[a_1,a_2,\dots,a_n]$ with $\s_{\leftrightarrow}$ its palindromification. Let $\Tilde{\s} = \s[a_2,\dots,a_n]$; then
    \begin{equation*}
        m(\s_{\leftrightarrow}) = m(\s)^2 + m(\Tilde{\s})^2.
    \end{equation*}
\end{theorem}
Consequently, we obtain the following corollary;
\begin{corollary}
    Let $[a_1,\dots,a_n] = p_n/q_n$; then
    \begin{equation*}
        [a_n,\dots,a_1,a_1,\dots,a_n] = \dfrac{p_n^2 + q_n^2}{p_{n-1}p_n + q_{n-1}q_n}.
    \end{equation*}
\end{corollary}
\begin{proof}
    Notice that by Theorem \ref{thm3.6}, we have;
    \begin{align}
        [a_n,\dots,a_1,a_1,\dots,a_n] &= \dfrac{m(\pal{\s})}{m(\s[a_{n-1},\dots,a_1,a_1,\dots,a_n])}
    \end{align}
    where via Theorem \ref{thm3.12} and \ref{thm3.11}, the left side becomes;
    \begin{equation*}
        \dfrac{m(\s)^2 + m(\Tilde{\s})^2}{m(\s[a_{n-1},\dots,a_1])m(\s[a_1,\dots,a_n])+ m(\s[a_{n-1},\dots,a_2])m(\s[a_2,\dots,a_n])};
    \end{equation*}
    which by symmetry it is equal to $ \dfrac{p_n^2 + q_n^2}{p_{n-1}p_n + q_{n-1}q_n}$.
\end{proof}
\begin{example}
    Consider the continued fraction $[3,1,5] = 23/6$; then observe that $[3,1] = 4$, and its palindromification
    \begin{equation*}
        [5,1,3,3,1,5] = \dfrac{565}{98}= \dfrac{23^2 + 6^2}{4 \cdot 23 + 1 \cdot 6}
    \end{equation*}.
\end{example}
Suppose that we have an integer $N$, such that we can write $N = p^2 + q^2$, where $p > q \geq 1$ such that gcd$(p,q) = 1$. Then we say that $N$ is a \emph{sum of two relatively prime squares}. Consequently, we obtain the following corollary;
\begin{corollary}~ 
    \begin{itemize}
        \item[(a)] If $N$ is a sum of two relatively prime squares, then there exists a palindromic snake graph of even length $\s$ such that $m(\s) = N$;
        \item[(b)] For each positive integer $N$, the number of ways one can write $N$ as a sum of two relatively prime squares is equal to is equal to half the number of palindromic snake graphs of even length with $N$ perfect matchings;
        \item[(c)] For each positive integer $N$, the number of ways one can write $N$ as a sum of two relatively prime squares is equal to half the number of palindromic continued fractions of even length with numerator equal to $N$. 
    \end{itemize}
\end{corollary}
\begin{proof}
    For part (a), suppose $p > q \geq 1$, with gcd$(p,q) = 1$; and let $[a_1,\dots,a_n] = p/q$. Then by Theorem \ref{thm3.12}, and Theorem \ref{thm3.6}, it follows that $\s[a_n,\dots,a_1,a_1,\dots,a_n]$ has precisely $N$ perfect matchings. For part (b) and (c), the bijections given in Theorem \ref{thm3.7} suffice.
\end{proof}
\chapter{Frobenius' Conjecture and Markov Numbers}
In number theory, more precisely, in the theory of Diophantine equations, one that is of particular interest is \emph{Markov's equation};
\begin{equation}\label{Markov}
    x^2 + y^2 + z^2 = 3xyz.
\end{equation}
A triple $(a,b,c), \ a \leq b \leq c$, that is a solution to (3.1) is called a \emph{Markov triple}, and $a,b,$ and $c$ are called \emph{Markov numbers}. A few of these are $(1,1,1),(1,1,2), (1,2,5), (1,5,13)$, $(1, 89, 233), (5, 29, 433)$. It is known that every other Fibonacci number is a Markov number, and so is every Pell number. The essence of Frobenius' conjecture is that of uniqueness of solutions;
\begin{conjecture}[Frobenius' Uniqueness Conjecture]\label{frobconj}
    Let $(a_1,a_2,\tau)$ and $(b_1,b_2,\tau)$ be Markov triples, then $a_1 = b_1$ and $a_2=b_2$. 
\end{conjecture}
In other words, Frobenius conjectured that every Markov triple is uniquely defined by its largest element. 
\section{Markov generating function}
We begin by considering the torus $\mathcal{T}^2$ as the quotient space
\begin{equation*}
    \mathcal{T}^2 \cong \mathcal{I}\times \mathcal{I}/\sim_{ns} \sim_{we},
\end{equation*}
where $\mathcal{I} = [0,1] \subseteq \mathbb{R}$, and $\sim_{ns}, \sim_{we}$ are the equivalence relations identifying \emph{north} with \emph{south} and \emph{west} with \emph{east}. Next, we triangulate it; which is much easier to do when viewing it via the quotient (as it is simply a diagonal) than as a 3-dimensional manifold; and then we remove a single point, more precisely the point $(0,0) \sim (0,1)\sim (1,0) \sim (1,1)$. In the figure below, we have the following image.
\begin{figure}[H]
    \centering
    

\tikzset{every picture/.style={line width=0.75pt}} %set default line width to 0.75pt        

\begin{tikzpicture}[x=0.75pt,y=0.75pt,yscale=-1,xscale=1,scale = 0.7]
%uncomment if require: \path (0,487); %set diagram left start at 0, and has height of 487

%Straight Lines [id:da8215288959193072] 
\draw    (100,121) -- (100.08,320) ;
%Straight Lines [id:da8526418629967348] 
\draw    (301,120) -- (300.08,320) ;
%Straight Lines [id:da5367620041050422] 
\draw    (100.08,320) -- (300.08,320) ;
%Straight Lines [id:da4784595305292364] 
\draw    (100,121) -- (300,121) ;
\draw   (93.79,213.01) -- (100.41,203.96) -- (106.89,213.11)(93.82,209.01) -- (100.44,199.96) -- (106.92,209.11) ;
\draw   (293.79,213.01) -- (300.41,203.96) -- (306.89,213.11)(293.82,209.01) -- (300.44,199.96) -- (306.92,209.11) ;
\draw   (195,114) -- (209,121) -- (195,128) ;
\draw   (197,313) -- (211,320) -- (197,327) ;
%Straight Lines [id:da22151170816098675] 
\draw    (100,121) -- (300.08,320) ;
\draw  [fill={rgb, 255:red, 0; green, 0; blue, 0 }  ,fill opacity=1 ] (92,121.25) .. controls (92,116.97) and (95.47,113.5) .. (99.75,113.5) .. controls (104.03,113.5) and (107.5,116.97) .. (107.5,121.25) .. controls (107.5,125.53) and (104.03,129) .. (99.75,129) .. controls (95.47,129) and (92,125.53) .. (92,121.25) -- cycle ; \draw   (92,121.25) -- (107.5,121.25) ; \draw   (99.75,113.5) -- (99.75,129) ;
\draw  [fill={rgb, 255:red, 5; green, 0; blue, 0 }  ,fill opacity=1 ] (293,121.25) .. controls (293,116.97) and (296.47,113.5) .. (300.75,113.5) .. controls (305.03,113.5) and (308.5,116.97) .. (308.5,121.25) .. controls (308.5,125.53) and (305.03,129) .. (300.75,129) .. controls (296.47,129) and (293,125.53) .. (293,121.25) -- cycle ; \draw   (293,121.25) -- (308.5,121.25) ; \draw   (300.75,113.5) -- (300.75,129) ;
\draw  [fill={rgb, 255:red, 8; green, 0; blue, 0 }  ,fill opacity=1 ] (292,320.25) .. controls (292,315.97) and (295.47,312.5) .. (299.75,312.5) .. controls (304.03,312.5) and (307.5,315.97) .. (307.5,320.25) .. controls (307.5,324.53) and (304.03,328) .. (299.75,328) .. controls (295.47,328) and (292,324.53) .. (292,320.25) -- cycle ; \draw   (292,320.25) -- (307.5,320.25) ; \draw   (299.75,312.5) -- (299.75,328) ;
\draw  [color={rgb, 255:red, 3; green, 0; blue, 0 }  ,draw opacity=1 ][fill={rgb, 255:red, 7; green, 0; blue, 0 }  ,fill opacity=1 ] (92,320.25) .. controls (92,315.97) and (95.47,312.5) .. (99.75,312.5) .. controls (104.03,312.5) and (107.5,315.97) .. (107.5,320.25) .. controls (107.5,324.53) and (104.03,328) .. (99.75,328) .. controls (95.47,328) and (92,324.53) .. (92,320.25) -- cycle ; \draw  [color={rgb, 255:red, 3; green, 0; blue, 0 }  ,draw opacity=1 ] (92,320.25) -- (107.5,320.25) ; \draw  [color={rgb, 255:red, 3; green, 0; blue, 0 }  ,draw opacity=1 ] (99.75,312.5) -- (99.75,328) ;




\end{tikzpicture}

    \label{Torus}
    \caption{Triangulated torus $\mathcal{T}^2$.}
\end{figure}
After we label each side (of which there are 3) we fix a \emph{clockwise orientation}; i.e. as we approach where two diagonals meet, the orientation is as follows;
\begin{figure}[H]
    \centering


    \tikzset{every picture/.style={line width=0.75pt}} %set default line width to 0.75pt        

    \begin{tikzpicture}[x=0.75pt,y=0.75pt,yscale=-1,xscale=1]
    %uncomment if require: \path (0,487); %set diagram left start at 0, and has height of 487
    
    %Straight Lines [id:da4370862379579007] 
    \draw    (120.17,250.5) -- (240.5,190.83) ;
    %Straight Lines [id:da8355747741827015] 
    \draw    (120.17,250.5) -- (239.5,249.83) ;
    %Curve Lines [id:da10366436706962945] 
    \draw [color={rgb, 255:red, 251; green, 0; blue, 0 }  ,draw opacity=1 ] [dash pattern={on 4.5pt off 4.5pt}]  (218.17,244.5) .. controls (202.1,239.78) and (201.22,227.01) .. (209.61,213.81) ;
    \draw [shift={(211.17,211.5)}, rotate = 125.54] [fill={rgb, 255:red, 251; green, 0; blue, 0 }  ,fill opacity=1 ][line width=0.08]  [draw opacity=0] (8.93,-4.29) -- (0,0) -- (8.93,4.29) -- cycle    ;
    
    
    
    
    \end{tikzpicture}

    
    
\end{figure}
If we then apply it to our construction, we obtain the following;

\begin{figure}[!htb]
\centering
\minipage[c]{0.375\textwidth}
\footnotesize


\tikzset{every picture/.style={line width=0.75pt}} %set default line width to 0.75pt        

\begin{tikzpicture}[x=0.75pt,y=0.75pt,yscale=-1,xscale=1]
%uncomment if require: \path (0,487); %set diagram left start at 0, and has height of 487

%Straight Lines [id:da8215288959193072] 
\draw    (100,121) -- (100.08,320) ;
%Straight Lines [id:da8526418629967348] 
\draw    (301,120) -- (300.08,320) ;
%Straight Lines [id:da5367620041050422] 
\draw    (100.08,320) -- (300.08,320) ;
%Straight Lines [id:da4784595305292364] 
\draw    (100,121) -- (300,121) ;
\draw   (93.79,213.01) -- (100.41,203.96) -- (106.89,213.11)(93.82,209.01) -- (100.44,199.96) -- (106.92,209.11) ;
\draw   (293.79,213.01) -- (300.41,203.96) -- (306.89,213.11)(293.82,209.01) -- (300.44,199.96) -- (306.92,209.11) ;
\draw   (195,114) -- (209,121) -- (195,128) ;
\draw   (197,313) -- (211,320) -- (197,327) ;
%Straight Lines [id:da22151170816098675] 
\draw    (100,121) -- (300.08,320) ;
\draw  [color={rgb, 255:red, 10; green, 0; blue, 0 }  ,draw opacity=1 ][fill={rgb, 255:red, 14; green, 1; blue, 1 }  ,fill opacity=1 ] (92,121.25) .. controls (92,116.97) and (95.47,113.5) .. (99.75,113.5) .. controls (104.03,113.5) and (107.5,116.97) .. (107.5,121.25) .. controls (107.5,125.53) and (104.03,129) .. (99.75,129) .. controls (95.47,129) and (92,125.53) .. (92,121.25) -- cycle ; \draw  [color={rgb, 255:red, 10; green, 0; blue, 0 }  ,draw opacity=1 ] (92,121.25) -- (107.5,121.25) ; \draw  [color={rgb, 255:red, 10; green, 0; blue, 0 }  ,draw opacity=1 ] (99.75,113.5) -- (99.75,129) ;
\draw  [fill={rgb, 255:red, 7; green, 0; blue, 0 }  ,fill opacity=1 ] (293,121.25) .. controls (293,116.97) and (296.47,113.5) .. (300.75,113.5) .. controls (305.03,113.5) and (308.5,116.97) .. (308.5,121.25) .. controls (308.5,125.53) and (305.03,129) .. (300.75,129) .. controls (296.47,129) and (293,125.53) .. (293,121.25) -- cycle ; \draw   (293,121.25) -- (308.5,121.25) ; \draw   (300.75,113.5) -- (300.75,129) ;
\draw  [fill={rgb, 255:red, 8; green, 0; blue, 0 }  ,fill opacity=1 ] (92,320.25) .. controls (92,315.97) and (95.47,312.5) .. (99.75,312.5) .. controls (104.03,312.5) and (107.5,315.97) .. (107.5,320.25) .. controls (107.5,324.53) and (104.03,328) .. (99.75,328) .. controls (95.47,328) and (92,324.53) .. (92,320.25) -- cycle ; \draw   (92,320.25) -- (107.5,320.25) ; \draw   (99.75,312.5) -- (99.75,328) ;
\draw  [fill={rgb, 255:red, 7; green, 0; blue, 0 }  ,fill opacity=1 ] (293,320.25) .. controls (293,315.97) and (296.47,312.5) .. (300.75,312.5) .. controls (305.03,312.5) and (308.5,315.97) .. (308.5,320.25) .. controls (308.5,324.53) and (305.03,328) .. (300.75,328) .. controls (296.47,328) and (293,324.53) .. (293,320.25) -- cycle ; \draw   (293,320.25) -- (308.5,320.25) ; \draw   (300.75,312.5) -- (300.75,328) ;

% Text Node
\draw (84,230) node [anchor=north west][inner sep=0.75pt]   [align=left] {1};
% Text Node
\draw (307,170) node [anchor=north west][inner sep=0.75pt]   [align=left] {1};
% Text Node
\draw (166,327) node [anchor=north west][inner sep=0.75pt]   [align=left] {2};
% Text Node
\draw (240,97) node [anchor=north west][inner sep=0.75pt]   [align=left] {2};
% Text Node
\draw (167,166) node [anchor=north west][inner sep=0.75pt]   [align=left] {3};


\end{tikzpicture}

\endminipage\hfill
\minipage[c]{0.25\textwidth}
\centering
\footnotesize
\tikzset{every picture/.style={line width=0.75pt}}
\begin{tikzpicture}[x=0.75pt,y=0.75pt,yscale=-1,xscale=1]
\draw    (105,190) .. controls (187.17,81.5) and (98.17,245.5) .. (230.17,182.5) ;
\draw [shift={(230.17,182.5)}, rotate = 154.49] [color={rgb, 255:red, 0; green, 0; blue, 0 }  ][line width=0.75]    (21.86,-6.58) .. controls (13.9,-2.79) and (6.61,-0.6) .. (0,0) .. controls (6.61,0.6) and (13.9,2.79) .. (21.86,6.58)   ;
\end{tikzpicture}
\endminipage\hfill
\minipage[c]{0.375\textwidth}%
\footnotesize


\tikzset{every picture/.style={line width=0.75pt}} %set default line width to 0.75pt        

\begin{tikzpicture}[x=0.75pt,y=0.75pt,yscale=-1,xscale=1]
%uncomment if require: \path (0,487); %set diagram left start at 0, and has height of 487

%Straight Lines [id:da8215288959193072] 
\draw    (100,121) -- (100.08,320) ;
%Straight Lines [id:da8526418629967348] 
\draw    (301,120) -- (300.08,320) ;
%Straight Lines [id:da5367620041050422] 
\draw    (100.08,320) -- (300.08,320) ;
%Straight Lines [id:da4784595305292364] 
\draw    (100,121) -- (300,121) ;
\draw   (93.79,213.01) -- (100.41,203.96) -- (106.89,213.11)(93.82,209.01) -- (100.44,199.96) -- (106.92,209.11) ;
\draw   (293.79,213.01) -- (300.41,203.96) -- (306.89,213.11)(293.82,209.01) -- (300.44,199.96) -- (306.92,209.11) ;
\draw   (195,114) -- (209,121) -- (195,128) ;
\draw   (197,313) -- (211,320) -- (197,327) ;
%Straight Lines [id:da22151170816098675] 
\draw    (100,121) -- (300.08,320) ;
\draw  [fill={rgb, 255:red, 0; green, 0; blue, 0 }  ,fill opacity=1 ] (92,121.25) .. controls (92,116.97) and (95.47,113.5) .. (99.75,113.5) .. controls (104.03,113.5) and (107.5,116.97) .. (107.5,121.25) .. controls (107.5,125.53) and (104.03,129) .. (99.75,129) .. controls (95.47,129) and (92,125.53) .. (92,121.25) -- cycle ; \draw   (92,121.25) -- (107.5,121.25) ; \draw   (99.75,113.5) -- (99.75,129) ;
\draw  [fill={rgb, 255:red, 0; green, 0; blue, 0 }  ,fill opacity=1 ] (293,121.25) .. controls (293,116.97) and (296.47,113.5) .. (300.75,113.5) .. controls (305.03,113.5) and (308.5,116.97) .. (308.5,121.25) .. controls (308.5,125.53) and (305.03,129) .. (300.75,129) .. controls (296.47,129) and (293,125.53) .. (293,121.25) -- cycle ; \draw   (293,121.25) -- (308.5,121.25) ; \draw   (300.75,113.5) -- (300.75,129) ;
\draw  [fill={rgb, 255:red, 10; green, 0; blue, 0 }  ,fill opacity=1 ] (92,320.25) .. controls (92,315.97) and (95.47,312.5) .. (99.75,312.5) .. controls (104.03,312.5) and (107.5,315.97) .. (107.5,320.25) .. controls (107.5,324.53) and (104.03,328) .. (99.75,328) .. controls (95.47,328) and (92,324.53) .. (92,320.25) -- cycle ; \draw   (92,320.25) -- (107.5,320.25) ; \draw   (99.75,312.5) -- (99.75,328) ;
\draw  [fill={rgb, 255:red, 7; green, 0; blue, 0 }  ,fill opacity=1 ] (293,320.25) .. controls (293,315.97) and (296.47,312.5) .. (300.75,312.5) .. controls (305.03,312.5) and (308.5,315.97) .. (308.5,320.25) .. controls (308.5,324.53) and (305.03,328) .. (300.75,328) .. controls (296.47,328) and (293,324.53) .. (293,320.25) -- cycle ; \draw   (293,320.25) -- (308.5,320.25) ; \draw   (300.75,312.5) -- (300.75,328) ;
%Curve Lines [id:da13696914158614193] 
\draw [color={rgb, 255:red, 249; green, 1; blue, 1 }  ,draw opacity=1 ] [dash pattern={on 4.5pt off 4.5pt}]  (114.42,234.54) .. controls (120.24,218.54) and (129.82,207.71) .. (157.77,195.66) ;
\draw [shift={(160.42,194.54)}, rotate = 157.38] [fill={rgb, 255:red, 249; green, 1; blue, 1 }  ,fill opacity=1 ][line width=0.08]  [draw opacity=0] (8.93,-4.29) -- (0,0) -- (8.93,4.29) -- cycle    ;
%Curve Lines [id:da42631957373483753] 
\draw [color={rgb, 255:red, 249; green, 1; blue, 1 }  ,draw opacity=1 ] [dash pattern={on 4.5pt off 4.5pt}]  (151.42,300.04) .. controls (135.9,290.83) and (127.43,279.26) .. (115.07,251.19) ;
\draw [shift={(113.92,248.54)}, rotate = 66.57] [fill={rgb, 255:red, 249; green, 1; blue, 1 }  ,fill opacity=1 ][line width=0.08]  [draw opacity=0] (8.93,-4.29) -- (0,0) -- (8.93,4.29) -- cycle    ;
%Curve Lines [id:da5983276540829184] 
\draw [color={rgb, 255:red, 249; green, 1; blue, 1 }  ,draw opacity=1 ] [dash pattern={on 4.5pt off 4.5pt}]  (180.92,211.04) .. controls (183.86,238.48) and (195.92,263.52) .. (172.4,297.45) ;
\draw [shift={(170.92,299.54)}, rotate = 306.08] [fill={rgb, 255:red, 249; green, 1; blue, 1 }  ,fill opacity=1 ][line width=0.08]  [draw opacity=0] (8.93,-4.29) -- (0,0) -- (8.93,4.29) -- cycle    ;
%Curve Lines [id:da9247996561877188] 
\draw [color={rgb, 255:red, 249; green, 1; blue, 1 }  ,draw opacity=1 ] [dash pattern={on 4.5pt off 4.5pt}]  (276.54,200.95) .. controls (268.99,216.21) and (258.27,225.92) .. (229.16,234.81) ;
\draw [shift={(226.41,235.63)}, rotate = 343.71] [fill={rgb, 255:red, 249; green, 1; blue, 1 }  ,fill opacity=1 ][line width=0.08]  [draw opacity=0] (8.93,-4.29) -- (0,0) -- (8.93,4.29) -- cycle    ;
%Curve Lines [id:da42565795945334595] 
\draw [color={rgb, 255:red, 249; green, 1; blue, 1 }  ,draw opacity=1 ] [dash pattern={on 4.5pt off 4.5pt}]  (246.99,131.77) .. controls (261.4,142.64) and (268.53,155.07) .. (277.72,184.32) ;
\draw [shift={(278.58,187.09)}, rotate = 252.9] [fill={rgb, 255:red, 249; green, 1; blue, 1 }  ,fill opacity=1 ][line width=0.08]  [draw opacity=0] (8.93,-4.29) -- (0,0) -- (8.93,4.29) -- cycle    ;
%Curve Lines [id:da31674209595039604] 
\draw [color={rgb, 255:red, 249; green, 1; blue, 1 }  ,draw opacity=1 ] [dash pattern={on 4.5pt off 4.5pt}]  (207.85,216.97) .. controls (207.96,189.37) and (198.73,163.16) .. (225.84,132.02) ;
\draw [shift={(227.55,130.11)}, rotate = 132.41] [fill={rgb, 255:red, 249; green, 1; blue, 1 }  ,fill opacity=1 ][line width=0.08]  [draw opacity=0] (8.93,-4.29) -- (0,0) -- (8.93,4.29) -- cycle    ;

% Text Node
\draw (84,230) node [anchor=north west][inner sep=0.75pt]   [align=left] {1};
% Text Node
\draw (307,170) node [anchor=north west][inner sep=0.75pt]   [align=left] {1};
% Text Node
\draw (166,327) node [anchor=north west][inner sep=0.75pt]   [align=left] {2};
% Text Node
\draw (240,97) node [anchor=north west][inner sep=0.75pt]   [align=left] {2};
% Text Node
\draw (190,196) node [anchor=north west][inner sep=0.75pt]   [align=left] {3};


\end{tikzpicture}

\endminipage
\end{figure}
Observe that now we have precisely 2 arrows $1 \rightarrow 3$, 2 arrows $3 \rightarrow 2$ and 2 arrows $2 \rightarrow 1$; which we can then us to construct the following quiver $\mathcal{Q}$;
\begin{figure}[H]
    \centering
   \begin{tikzcd}
	&& 2 \\
	\\
	\\
	1 &&&& 3
	\arrow[shift left=1, from=1-3, to=4-1]
	\arrow[shift left=1, from=4-1, to=4-5]
	\arrow[shift left=1, from=4-5, to=1-3]
	\arrow[shift right=1, from=1-3, to=4-1]
	\arrow[shift right=1, from=4-1, to=4-5]
	\arrow[shift right=1, from=4-5, to=1-3]
\end{tikzcd}
\end{figure}
This is also known as the \emph{Markov quiver}. Let $\mathbf{x} = (x_1,x_2,x_3)$, and $\mathbf{y} = (1,1,1)$; then define the seed $(\mathbf{x},\mathbf{y},\mathcal{Q})$ and consider the mutation $\mu_1$.\footnote{Vertex 1 is obviously arbitrary, and we could have taken $\mu_2$ or $\mu_3$ without yielding \emph{meaningfully} different results.} Recall that since $\mathbf{y}$ is a vector of 1's, we can leave it out throughout our calculations. Consequently, we obtain that $x_1^{'} = (x_2^2 + x_3^2)/x_1$; and through (\ref{Markov}), $x_1^{'} = 3x_2x_3 - x_1$; i.e. $\mu_1$ acts on a triple $(x,y,z)$ by 
\begin{equation}
    (x,y,z) \xrightarrow{\mu_1} (3yz-x,y,z).
\end{equation}
Observe that, given $x\leq y \leq z$ such that $x^2 + y^2 +z^2 = 3xyz$; i.e. $(x,y,z)$ is a Markov triple, then
\begin{align*}
    (3yz-x)^2 + y^2 + z^2 &= 9y^2z^2 - 6xyz + x^2 + y^2 + z^2 \\
    &= 9y^2z^2 - 6xyz + 3xyz \\
    &= 9y^2z^2-3xyz \\
    &= 3(3yz-x)yz.
\end{align*}
Thus, we see that that if $(x,y,z)$ is a Markov triple then $\mu_1(x,y,z) = (3yz-x,y,z)$ is also a Markov triple.

For example, begin with $(x,y,z) = (1,1,1)$, then
\begin{align*}
    \mu_1(1,1,1) &= (2,1,1) \sim (1,1,2), \\
    \mu_1(1,1,2) &= (5,1,2) \sim (1,2,5), \\
    \mu_1(1,2,5) &= (29,2,5) \sim (2,5,29), \\
    \mu_1(2,5,29) &= (433,5,29) \sim (5,29,433), \\
    & \ \vdots
\end{align*}
If we apply it to all Markov triples, we can construct a branch of the \emph{Markov Number Tree};
\begin{figure}[H]
    \centering
    

\tikzset{every picture/.style={line width=0.75pt}} %set default line width to 0.75pt        

\begin{tikzpicture}[x=0.75pt,y=0.75pt,yscale=-1,xscale=1,scale = 0.95]
%uncomment if require: \path (0,487); %set diagram left start at 0, and has height of 487

%Straight Lines [id:da218791233847758] 
\draw    (57.17,201.5) -- (104.17,201.5) ;
%Straight Lines [id:da8091347234992389] 
\draw    (145.17,200.5) -- (195.17,200.5) ;
%Straight Lines [id:da03319795115091129] 
\draw    (240.17,206.5) -- (294.17,270.5) ;
%Straight Lines [id:da198756824632496] 
\draw    (240.17,194.5) -- (296.17,132.5) ;
%Straight Lines [id:da15958487534659938] 
\draw    (330.17,121.5) -- (370.17,91.5) ;
%Straight Lines [id:da7930368539552297] 
\draw    (330.17,121.5) -- (374.17,149.5) ;
%Straight Lines [id:da32547626002854535] 
\draw    (332.17,281.5) -- (372.17,251.5) ;
%Straight Lines [id:da7210342147933023] 
\draw    (332.17,281.5) -- (376.17,309.5) ;
%Straight Lines [id:da3606890158328505] 
\draw    (412.17,241.5) -- (452.17,250.5) ;
%Straight Lines [id:da42264726338809444] 
\draw    (412.17,241.5) -- (467.17,230.5) ;
%Straight Lines [id:da42552194803705357] 
\draw    (412.17,162.5) -- (452.17,171.5) ;
%Straight Lines [id:da14439933873063715] 
\draw    (412.17,162.5) -- (467.17,151.5) ;
%Straight Lines [id:da6527603824078907] 
\draw    (407.17,82.5) -- (447.17,91.5) ;
%Straight Lines [id:da7855197344437875] 
\draw    (407.17,82.5) -- (462.17,71.5) ;
%Straight Lines [id:da9841190877605993] 
\draw    (417.17,321.5) -- (457.17,330.5) ;
%Straight Lines [id:da05891146298264216] 
\draw    (417.17,321.5) -- (472.17,310.5) ;

% Text Node
\draw (13,192) node [anchor=north west][inner sep=0.75pt]   [align=left] {(1,1,1)};
% Text Node
\draw (102,193) node [anchor=north west][inner sep=0.75pt]   [align=left] {(1,1,2)};
% Text Node
\draw (193,192) node [anchor=north west][inner sep=0.75pt]   [align=left] {(1,2,5)};
% Text Node
\draw (272,113) node [anchor=north west][inner sep=0.75pt]   [align=left] {(1,5,13)};
% Text Node
\draw (272,272) node [anchor=north west][inner sep=0.75pt]   [align=left] {(2,5,29)};
% Text Node
\draw (341,233) node [anchor=north west][inner sep=0.75pt]   [align=left] {(5,29,433)};
% Text Node
\draw (342,314) node [anchor=north west][inner sep=0.75pt]   [align=left] {(2,29,169)};
% Text Node
\draw (338,153) node [anchor=north west][inner sep=0.75pt]   [align=left] {(5,13,194)};
% Text Node
\draw (342,71) node [anchor=north west][inner sep=0.75pt]   [align=left] {(1,13,34)};
% Text Node
\draw (424,252) node [anchor=north west][inner sep=0.75pt]   [align=left] {(29,433,37666)};
% Text Node
\draw (439,332) node [anchor=north west][inner sep=0.75pt]   [align=left] {(2,169,985)};
% Text Node
\draw (429,173) node [anchor=north west][inner sep=0.75pt]   [align=left] {(5,194,2897)};
% Text Node
\draw (424,132) node [anchor=north west][inner sep=0.75pt]   [align=left] {(13,194,7651)};
% Text Node
\draw (432,210) node [anchor=north west][inner sep=0.75pt]   [align=left] {(5,433,6466)};
% Text Node
\draw (425,291) node [anchor=north west][inner sep=0.75pt]   [align=left] {(29,169,14701)};
% Text Node
\draw (423,93) node [anchor=north west][inner sep=0.75pt]   [align=left] {(13,34,1325)};
% Text Node
\draw (439,50) node [anchor=north west][inner sep=0.75pt]   [align=left] {(1,34,89)};
% Text Node
\draw (172,149) node [anchor=north west][inner sep=0.75pt]   [align=left] {\textcolor[rgb]{0.96,0.05,0.05}{1}};
% Text Node
\draw (232,239) node [anchor=north west][inner sep=0.75pt]   [align=left] {\textcolor[rgb]{0.96,0,0}{2}};
% Text Node
\draw (299,190) node [anchor=north west][inner sep=0.75pt]   [align=left] {\textcolor[rgb]{0.96,0.02,0.02}{5}};
% Text Node
\draw (378,109) node [anchor=north west][inner sep=0.75pt]   [align=left] {\textcolor[rgb]{0.98,0.02,0.02}{13}};
% Text Node
\draw (464.17,74.5) node [anchor=north west][inner sep=0.75pt]   [align=left] {\textcolor[rgb]{0.96,0,0}{34}};
% Text Node
\draw (469.17,154.5) node [anchor=north west][inner sep=0.75pt]   [align=left] {\textcolor[rgb]{0.96,0,0}{194}};
% Text Node
\draw (469.17,233.5) node [anchor=north west][inner sep=0.75pt]   [align=left] {\textcolor[rgb]{0.96,0,0}{433}};
% Text Node
\draw (474.17,313.5) node [anchor=north west][inner sep=0.75pt]   [align=left] {\textcolor[rgb]{0.96,0,0}{169}};
% Text Node
\draw (377,272) node [anchor=north west][inner sep=0.75pt]   [align=left] {\textcolor[rgb]{0.96,0,0}{29}};


\end{tikzpicture}

\end{figure}
 The same process can be done to yield two the other mutations $\mu_2,\mu_3$. Similarly, for all three we have that if we start with a Markov triple $(x,y,z)$, then $\mu_i(x,y,z)$ is also a Markov triple. However, $\mu_i(x,y,z)$ need not be equal to $\mu_j(x,y,z)$, for $i \ne j$. For example, if we take $\mu_2$ (the reader might like to prove that $(x,y,z) \xrightarrow{\mu_2} (x,3xz-y,z)$) we can see that;
\begin{equation*}
    \mu_2(1,2,5) = (1,13,5) \sim (1,5,13) \ne (2,5,19) = \mu_1(1,2,5).
\end{equation*}
Meaning that we can go from triple to triple, in the Markov tree, by a sequence of these mutations. Additionally, the resulting quiver, after any of these mutations, becomes;
\begin{figure}[H]
    \centering
    \[\begin{tikzcd}
	&& 2\\
	\\
	\\
	1 &&&& 3
	\arrow[shift left=1, from=4-1, to=1-3]
	\arrow[shift right=1, from=4-1, to=1-3]
	\arrow[shift left=1, from=1-3, to=4-5]
	\arrow[shift right=1, from=1-3, to=4-5]
	\arrow[shift left=1, from=4-5, to=4-1]
	\arrow[shift right=1, from=4-5, to=4-1]
\end{tikzcd}\]
\end{figure}
which is clearly just the initial Markov quiver simply with all arrows inverted. This yields the following corollary;
\begin{corollary}
The Markov quiver has a single equivalence class with respect to cluster mutations.
\end{corollary}

\section{Markov Numbers}
As seen in the previous section, Markov triples are related to the clusters of the cluster algebra corresponding to the once-punctured torus. As the cluster variables are computed by snake graphs, we can view Markov numbers in terms of snake graphs. First off, begin by considering the natural number lattice $\mathbb{N} \times \mathbb{N}$, \\

\begin{figure}[H]
    \centering
\begin{tikzpicture}[scale = 0.6]
\begin{pgfonlayer}{bg}
    
    
    \clip (0,0) rectangle (7cm,7cm);
    \coordinate (Origin)   at (0,0);
    \coordinate (XAxisMin) at (0,0);
    \coordinate (XAxisMax) at (7,0);
    \coordinate (YAxisMin) at (0,0);
    \coordinate (YAxisMax) at (0,7);


     
    \draw[style=help lines] (-14,-14) grid[step=1cm] (14,14);
\end{pgfonlayer}   

    \draw [very thick, black,-latex] (XAxisMin) -- (XAxisMax);% Draw x axis
     \draw [very thick, black,-latex] (YAxisMin) -- (YAxisMax);% Draw y axis


    \foreach \x in {0,1,...,7}{                           % Two indices running over each
    \foreach \y in {0,1,...,7}{                       % node on the grid we have drawn 
    \node[draw,circle,inner sep=0.6pt,fill] at (\x,\y) {}; % Places a dot at those points

    %\foreach \a in {0,1,...,6}{
    %\foreach \b in {0,1,...,6}{
    %\draw[dashed, gray, very thin] (\a, \b+1) -- (\a+1,\b);
}
}


\end{tikzpicture}
\caption{Natural number lattice $\mathbb{N} \times \mathbb{N}$.}\label{natlattice}
\end{figure}
Suppose we now pick a slope $p/q$ where gcd$(p,q) = 1$ and $p < q$; then there is an associated Markov number $m_{p/q}$; which is exactly the number of terms in the numerator of the cluster variable represented by the line segment from $(0,0)$ to $(q,p)$. For example, consider the fraction $4/9$.  
\begin{figure}[H]\label{ChristoffelPathSlope}
    \centering
\begin{tikzpicture}
\begin{pgfonlayer}{bg}
    \clip (0,0) rectangle (9cm,4cm);
    \draw[style=help lines] (-14,-14) grid[step=1cm] (14,14);
    \end{pgfonlayer}
    \newcommand{\Square}[2]{\draw[thick] (#1,#2) -- (#1,#2 + 0.5) -- (#1 + 0.5,#2 + 0.5) -- (#1+0.5,#2)};


    
    \node[draw,circle,inner sep=1.5pt,fill] at (0, 0) {};
    \draw[style = help lines, color = blue, very thick] (0,0) -- (1,0);
    \Square{0.5}{0};
    \node[draw,circle,inner sep=1.5pt,fill] at (1, 0) {};
    
    \draw[style = help lines, color = blue, very thick] (1,0) -- (2,0);
    \node[draw,circle,inner sep=1.5pt,fill] at (2, 0) {};
    \draw[style = help lines, color = blue, very thick] (2,0) -- (3,0);
    \node[draw,circle,inner sep=1.5pt,fill] at (3, 0) {};
    \draw[style = help lines, color = blue, very thick] (3,0) -- (3,1);
    \node[draw,circle,inner sep=1.5pt,fill] at (3, 1) {};
    \draw[style = help lines, color = blue, very thick] (3,1) -- (4,1);
    \node[draw,circle,inner sep=1.5pt,fill] at (4, 1) {};
    \draw[style = help lines, color = blue, very thick] (4,1) -- (5,1);
    \node[draw,circle,inner sep=1.5pt,fill] at (5, 1) {};
    \draw[style = help lines, color = blue, very thick] (5,1) -- (5,2);
    \node[draw,circle,inner sep=1.5pt,fill] at (5, 2) {};
    \draw[style = help lines, color = blue, very thick] (5,2) -- (6,2);
    \node[draw,circle,inner sep=1.5pt,fill] at (6, 2) {};
    \draw[style = help lines, color = blue, very thick] (6,2) -- (7,2);
    \node[draw,circle,inner sep=1.5pt,fill] at (7, 2) {};
    \draw[style = help lines, color = blue, very thick] (7,2) -- (7,3);
    \node[draw,circle,inner sep=1.5pt,fill] at (7, 3) {};
    \draw[style = help lines, color = blue, very thick] (7,3) -- (8,3);
    \node[draw,circle,inner sep=1.5pt,fill] at (8, 3) {};
    \draw[style = help lines, color = blue, very thick] (8,3) -- (9,3);
    \node[draw,circle,inner sep=1.5pt,fill] at (9, 3) {};
    \draw[style = help lines, color = blue, very thick] (9,3) -- (9,4);
    \node[draw,circle,inner sep=1.5pt,fill, label=right : {\small (9,4)}] at (9 , 4) {};
    \draw[style = help lines, color = red] (0,0) -- (9,4);

    \Square{1}{0};
    \Square{1.5}{0};
    \Square{2}{0};
    \Square{2.5}{0};
    \Square{2.5}{0.5};
    \Square{2.5}{1};
    \Square{3}{1};
    \Square{3.5}{1};
    \Square{4}{1};
    \Square{4.5}{1};
    \Square{4.5}{1.5};
    \Square{4.5}{2};
    \Square{5}{2};
    \Square{5.5}{2};
    \Square{6}{2};
    \Square{6.5}{2};
    \Square{6.5}{2.5};
    \Square{6.5}{3};
    \Square{7}{3};
    \Square{7.5}{3};
    \Square{8}{3};
    \Square{8.5}{3};
\end{tikzpicture}
\caption{The slope $4/9$ with its \emph{lower Christoffel path} with  in blue, and its corresponding snake graph (called a \emph{Markov snake graph}) obtained by placing half unit squares on the Christoffel path leaving the first and last steps empty.}
\end{figure}
In Figure \ref{ChristoffelPathSlope}, we see that the continued fraction corresponding to the snake graph is precisely the palindromic continued fraction of even length $[2,1,1,2,2,1,1,2,2,1,1,2,2,1,1,2]$; which corresponds to the fraction $43261/16725$. One can check that its numerator, $43261$, is a Markov number; more precisely, it belongs to the Markov triple $(5,2897,43261)$. In other words, we have that $m_{4/9} = 43261$. 

For arbitrary $p,q$, the continued fraction corresponding to the snake graph as we constructed above is precisely
\begin{equation*}
        \begin{array}{cccccccccccccccccccccccc}
  [2,& \underbrace{1,\dots,1},&  2,& 2,&  \underbrace{1,\dots,1},& 2,& 2, & \cdots & 2,& 2,& \underbrace{1,\dots,1},& 2] ;  \\
 &2(v_1 -1) & & &2(v_2-1)& & &\ldots& & &2(v_p-1) & 
\end{array}
    \end{equation*}
where $v_i$, for all $i = 1,\dots,p$, is calculated as follows;
\begin{align*}
    v_1 &= \left\lfloor \dfrac{q}{p}\right\rfloor; \\
    v_i &= \left\lfloor \dfrac{iq}{p}\right\rfloor - \sum_{j=1}^{i-1}v_j, \ \text{for } i = 1,\dots,p-1; \\
    v_p &= q - 1 - \sum_{j=1}^{p-1} v_j.
\end{align*}
Through the above, we may now formulate the following result, which was proven by Frobenius;
\begin{theorem}[\cite{F}, Section 10]\label{MarkovThm}
Every Markov number $m_{p/q}$ is the numerator of a palindromic continued fraction $[a_n,\dots,a_2,a_1,a_1,a_2,\dots,a_n]$ of even length such that;
\begin{enumerate}
    \item $a_i \in \{1,2\}, \ a_n = 2$; 
    \item If $p+1 = q$, then $n = p$ and $a_i = 2$ for all $i$; 
    \item If $p+1 < q$ then $\dfrac{c-1}{c} < \dfrac{p}{q} < \dfrac{c}{c+1}$, for a unique positive integer $c$ and;
    \begin{itemize}
        \item[(i)] there are at most $p+1$ subsequences of $2$s; the first and last are of odd length $2c-1$ and all others are of even length $2c$ or $2c+2$;
        \item[(ii)] there are at most $p$ maximal subsequences of $1$s; each of which is of even length $2(v_i-1)$ and $|v_i - v_j| \leq 1$ for all $i \neq j$.
    \end{itemize}
\end{enumerate}
Moreover, the resulting map
\begin{equation*}
    p/q \mapsto [a_n,\dots,a_2,a_1,a_1,a_2,\dots,a_n]
\end{equation*}
    from rational numbers between $0$ and $1$ to palindromic continued fractions of even length is injective.
\end{theorem}
Via the above theorem, together with Corollary \ref{cor3.10}, we obtain the following result;
\begin{corollary}
    Every Markov number (except for $1$ and $2$) is the sum of two relatively prime squares.
\end{corollary}
Note that given any Markov number $m$, the decomposition described above need not be unique. For example, consider the Markov number $610$, and notice that $610 = 23^2 + 9^2$ or $610 = 21^2 + 13^2$. Moreover, notice that $21/13 = [1,1,1,1,1,2]$, with palindromification $[2,1,1,1,1,1,1,1,1,1,1,2]$; which corresponds to the slope $1/7$; i.e. $m_{1/7} = 610$. On the other hand, $23/9 = [2,1,1,4]$, with palindromification $[4,1,1,2,2,1,1,4]$, which is not of Markov type. This yields the following corollary;
\begin{corollary}
    Let $m>2$ be a Markov number. Then there exist positive integers $a<b$ with gcd$(a,b) = 1$ such that $m = a^2 + b^2$, $2a \leq b < 3a$ and
    \begin{itemize}
        \item[(i)] the palindromification of the snake graph $\s(b/a)$\footnote{If $[a_1,\dots,a_n]$ is the continued fraction corresponding to $b/a$, then $\s(b/a) = \s[a_1,\dots,a_n]$.} is of Markov type;
        \item[(ii)] the continued fraction expansion of $b/a$ consists entirely of $1$s and $2$s.
    \end{itemize}
\end{corollary}
\begin{proof}
    Let $p/q$ be a slope such that $m = m_{p/q}$, and let $[a_n,\dots,a_1,a_1,\dots,a_n]$ be its corresponding continued fraction via the map in Theorem \ref{MarkovThm}; more precisely, $m$ is the numerator of the continued fraction above, and the snake graph $\s[a_n,\dots,a_1,a_1,\dots,a_n]$ is of Markov type. Next, let $b/a = [a_1,\dots,a_n]$ with $0<a<b$ and gcd$(a,b) = 1$; then as $a_1= 2$, we see that $2a \leq b < 3a$; where $2a = b \iff a = 1 \implies m = 5$. By Corollary \ref{cor3.10}, we have that $m = a^2 + b^2$; which proves part (i). Moreover, by the injective map in Theorem \ref{MarkovThm}, any Markov snake graph has a corresponding continued fraction consisting entirely of $1$s and $2$s. This proves part (ii). 
\end{proof}
The authors in \cite{CS1} conjectured that the pair $(a,b)$ in the above corollary is uniquely determined by the Markov number. Which leads us to the following conjecture;
\begin{conjecture}\label{firstconj}
    Let $m>2$ be a Markov number. Then there exist \textbf{unique} positive integers $a<b$ with gcd$(a,b)=1$ such that $m^2 = a^2 + b^2$, $2a \leq b < 3a$ and the palindromification of the snake graph $\mathcal{S}(b/a)$ is a Markov snake graph.  
\end{conjecture}
The following statement is stronger.
\begin{conjecture}\label{secondconj}
    Let $m>2$ be a Markov number. Then there exist unique positive integers $a<b$ with gcd$(a,b) = 1$ such that $m^2 = a^2 + b^2$, $2a \leq b < 3a$ and the palindromification of the snake graph $\s(b/a)$ contains only $1$s and $2$s.
\end{conjecture}
These conjectures have both been checked via computer for all Markov numbers of slope $p/q$ with $p < q < 70$; which are precisely 1493 numbers. Moreover, we have the following result from \cite{CS1}.
\begin{theorem}~
    \begin{itemize}
        \item[(a)] Conjecture \ref{secondconj} implies Conjecture \ref{firstconj}.
        \item[(b)] Conjecture \ref{firstconj} is equivalent to Conjecture \ref{frobconj}.
    \end{itemize}
\end{theorem}
Ultimately, we obtained that Conjecture 4.6 is essentially a reformulation of Frobenius' conjecture in Cluster algebraic terms. This has great ramifications in the study of Markov's equation as it allows the problem to be attacked from a different angle. 
\section{Ordering of Markov numbers}
Consider the triangulated natural number lattice, where we have a diagonal from the south-east corner to the north-west corner denoted by 3; in other words we have that every lattice square in \ref{natlattice} is replaces by a labeled triangulated torus. Suppose we have two lattice points $A$ and $B$, and a line $l_{AB}$ from $A$ to $B$; it follows that the line has slope $r/s$, where $(s,r)$ is the point $B-A$. Recall that if gcd$(r,s) =  1$, then this line does not intersect any other lattice point. 

Say we have the contrary, so $r/s$ reduces to a simplified fraction $p/q$ and gcd$(p,q)=1$; then the line $l_{ab}$ intersects precisely $t+1$ lattice points, where $t =$gcd$(r,s)$; denote these by $P_0 = A,P_1,\dots, P_t = B$, and note that $P_i = A + (q,p)i$, for $i = 0,1\dots,t$. Since crossing lattice points in the integer plane, implies that the corresponding arc on the triangulated torus intersects multiple vertices; which we want to avoid as causes issues. Therefore, we define the concepts of a \emph{left} and \emph{right deformation} of a line $l_{AB}$, denoted $\gamma_{AB}^L$ and $\gamma_{AB}^R$ respectively.
\begin{definition}
    A \emph{left deformation} $\gamma_{AB}^L$ of the line $l_{AB}$ is an infinitesimal deformation of $l_{AB}$ passing on the left of the points $P_0,P_1,\dots,P_t$.
\end{definition}
\begin{figure}[H]
    \centering



\tikzset{every picture/.style={line width=0.75pt}} %set default line width to 0.75pt        

\begin{tikzpicture}[x=0.75pt,y=0.75pt,yscale=-1,xscale=1,scale = 0.9]
%uncomment if require: \path (0,300); %set diagram left start at 0, and has height of 300

%Straight Lines [id:da6368538487564617] 
\draw    (146.36,238.98) -- (215.92,71.85) ;
%Shape: Ellipse [id:dp06152859270761557] 
\draw  [fill={rgb, 255:red, 6; green, 0; blue, 0 }  ,fill opacity=1 ] (141.57,238.98) .. controls (141.57,236.33) and (143.72,234.19) .. (146.36,234.19) .. controls (149.01,234.19) and (151.15,236.33) .. (151.15,238.98) .. controls (151.15,241.62) and (149.01,243.77) .. (146.36,243.77) .. controls (143.72,243.77) and (141.57,241.62) .. (141.57,238.98) -- cycle ;
%Shape: Ellipse [id:dp6322236117672533] 
\draw  [fill={rgb, 255:red, 6; green, 0; blue, 0 }  ,fill opacity=1 ] (159.95,195.67) .. controls (159.95,193.02) and (162.09,190.88) .. (164.74,190.88) .. controls (167.38,190.88) and (169.53,193.02) .. (169.53,195.67) .. controls (169.53,198.31) and (167.38,200.46) .. (164.74,200.46) .. controls (162.09,200.46) and (159.95,198.31) .. (159.95,195.67) -- cycle ;
%Shape: Circle [id:dp6288082749109036] 
\draw  [fill={rgb, 255:red, 6; green, 0; blue, 0 }  ,fill opacity=1 ] (175.7,156.3) .. controls (175.7,153.65) and (177.84,151.51) .. (180.49,151.51) .. controls (183.13,151.51) and (185.28,153.65) .. (185.28,156.3) .. controls (185.28,158.94) and (183.13,161.09) .. (180.49,161.09) .. controls (177.84,161.09) and (175.7,158.94) .. (175.7,156.3) -- cycle ;
%Shape: Ellipse [id:dp05459211000933106] 
\draw  [fill={rgb, 255:red, 6; green, 0; blue, 0 }  ,fill opacity=1 ] (194.07,114.3) .. controls (194.07,111.66) and (196.21,109.51) .. (198.86,109.51) .. controls (201.5,109.51) and (203.65,111.66) .. (203.65,114.3) .. controls (203.65,116.95) and (201.5,119.09) .. (198.86,119.09) .. controls (196.21,119.09) and (194.07,116.95) .. (194.07,114.3) -- cycle ;
%Shape: Ellipse [id:dp036093723387683796] 
\draw  [fill={rgb, 255:red, 6; green, 0; blue, 0 }  ,fill opacity=1 ] (211.13,71.85) .. controls (211.13,69.2) and (213.27,67.06) .. (215.92,67.06) .. controls (218.56,67.06) and (220.71,69.2) .. (220.71,71.85) .. controls (220.71,74.49) and (218.56,76.64) .. (215.92,76.64) .. controls (213.27,76.64) and (211.13,74.49) .. (211.13,71.85) -- cycle ;
%Shape: Arc [id:dp5617942386186798] 
\draw  [draw opacity=0] (175.34,161.74) .. controls (174.58,161.78) and (173.8,161.69) .. (173.03,161.46) .. controls (169.48,160.38) and (167.47,156.63) .. (168.55,153.08) .. controls (169.63,149.53) and (173.38,147.52) .. (176.93,148.6) .. controls (177.7,148.83) and (178.4,149.2) .. (179.01,149.65) -- (174.98,155.03) -- cycle ; \draw  [color={rgb, 255:red, 255; green, 0; blue, 0 }  ,draw opacity=1 ] (175.34,161.74) .. controls (174.58,161.78) and (173.8,161.69) .. (173.03,161.46) .. controls (169.48,160.38) and (167.47,156.63) .. (168.55,153.08) .. controls (169.63,149.53) and (173.38,147.52) .. (176.93,148.6) .. controls (177.7,148.83) and (178.4,149.2) .. (179.01,149.65) ;  
%Shape: Arc [id:dp08688131382457276] 
\draw  [draw opacity=0] (158.94,200.45) .. controls (158.18,200.49) and (157.39,200.41) .. (156.62,200.17) .. controls (153.07,199.09) and (151.07,195.34) .. (152.15,191.79) .. controls (153.22,188.24) and (156.98,186.24) .. (160.53,187.31) .. controls (161.3,187.55) and (162,187.91) .. (162.61,188.37) -- (158.57,193.74) -- cycle ; \draw  [color={rgb, 255:red, 255; green, 0; blue, 0 }  ,draw opacity=1 ] (158.94,200.45) .. controls (158.18,200.49) and (157.39,200.41) .. (156.62,200.17) .. controls (153.07,199.09) and (151.07,195.34) .. (152.15,191.79) .. controls (153.22,188.24) and (156.98,186.24) .. (160.53,187.31) .. controls (161.3,187.55) and (162,187.91) .. (162.61,188.37) ;  
%Shape: Arc [id:dp4304883357709586] 
\draw  [draw opacity=0] (193.72,119.74) .. controls (192.95,119.79) and (192.17,119.7) .. (191.4,119.46) .. controls (187.85,118.38) and (185.84,114.63) .. (186.92,111.08) .. controls (188,107.53) and (191.75,105.53) .. (195.3,106.61) .. controls (196.08,106.84) and (196.77,107.2) .. (197.38,107.66) -- (193.35,113.03) -- cycle ; \draw  [color={rgb, 255:red, 255; green, 0; blue, 0 }  ,draw opacity=1 ] (193.72,119.74) .. controls (192.95,119.79) and (192.17,119.7) .. (191.4,119.46) .. controls (187.85,118.38) and (185.84,114.63) .. (186.92,111.08) .. controls (188,107.53) and (191.75,105.53) .. (195.3,106.61) .. controls (196.08,106.84) and (196.77,107.2) .. (197.38,107.66) ;  
%Shape: Arc [id:dp8293831770315049] 
\draw  [draw opacity=0] (185.5,150.27) .. controls (186.26,150.21) and (187.05,150.27) .. (187.83,150.48) .. controls (191.41,151.44) and (193.53,155.13) .. (192.57,158.71) .. controls (191.6,162.3) and (187.92,164.42) .. (184.33,163.45) .. controls (183.55,163.24) and (182.84,162.91) .. (182.22,162.47) -- (186.08,156.97) -- cycle ; \draw  [color={rgb, 255:red, 17; green, 10; blue, 255 }  ,draw opacity=1 ] (185.5,150.27) .. controls (186.26,150.21) and (187.05,150.27) .. (187.83,150.48) .. controls (191.41,151.44) and (193.53,155.13) .. (192.57,158.71) .. controls (191.6,162.3) and (187.92,164.42) .. (184.33,163.45) .. controls (183.55,163.24) and (182.84,162.91) .. (182.22,162.47) ;  
%Shape: Arc [id:dp9112487143667883] 
\draw  [draw opacity=0] (203.29,108.43) .. controls (204.05,108.37) and (204.84,108.43) .. (205.62,108.64) .. controls (209.2,109.6) and (211.32,113.29) .. (210.36,116.87) .. controls (209.39,120.46) and (205.71,122.58) .. (202.12,121.61) .. controls (201.35,121.4) and (200.63,121.07) .. (200.01,120.63) -- (203.87,115.13) -- cycle ; \draw  [color={rgb, 255:red, 17; green, 10; blue, 255 }  ,draw opacity=1 ] (203.29,108.43) .. controls (204.05,108.37) and (204.84,108.43) .. (205.62,108.64) .. controls (209.2,109.6) and (211.32,113.29) .. (210.36,116.87) .. controls (209.39,120.46) and (205.71,122.58) .. (202.12,121.61) .. controls (201.35,121.4) and (200.63,121.07) .. (200.01,120.63) ;  
%Shape: Arc [id:dp9693832414850533] 
\draw  [draw opacity=0] (169.79,190.2) .. controls (170.55,190.14) and (171.33,190.2) .. (172.11,190.41) .. controls (175.69,191.38) and (177.81,195.06) .. (176.85,198.65) .. controls (175.89,202.23) and (172.2,204.35) .. (168.62,203.39) .. controls (167.84,203.18) and (167.13,202.84) .. (166.5,202.4) -- (170.36,196.9) -- cycle ; \draw  [color={rgb, 255:red, 17; green, 10; blue, 255 }  ,draw opacity=1 ] (169.79,190.2) .. controls (170.55,190.14) and (171.33,190.2) .. (172.11,190.41) .. controls (175.69,191.38) and (177.81,195.06) .. (176.85,198.65) .. controls (175.89,202.23) and (172.2,204.35) .. (168.62,203.39) .. controls (167.84,203.18) and (167.13,202.84) .. (166.5,202.4) ;  
%Straight Lines [id:da5107941415970204] 
\draw [color={rgb, 255:red, 38; green, 0; blue, 255 }  ,draw opacity=1 ]   (150.04,234.84) -- (166.5,202.4) ;
%Straight Lines [id:da46687813887134577] 
\draw [color={rgb, 255:red, 38; green, 0; blue, 255 }  ,draw opacity=1 ]   (169.79,190.2) -- (182.22,162.47) ;
%Straight Lines [id:da40818103901933445] 
\draw [color={rgb, 255:red, 38; green, 0; blue, 255 }  ,draw opacity=1 ]   (185.5,150.27) -- (200.01,120.63) ;
%Straight Lines [id:da5750863983302763] 
\draw [color={rgb, 255:red, 38; green, 0; blue, 255 }  ,draw opacity=1 ]   (203.29,108.43) -- (215.92,76.64) ;
%Straight Lines [id:da43400173674305065] 
\draw [color={rgb, 255:red, 255; green, 0; blue, 4 }  ,draw opacity=1 ]   (146.36,234.19) -- (158.94,200.45) ;
%Straight Lines [id:da3533414358041308] 
\draw [color={rgb, 255:red, 255; green, 0; blue, 4 }  ,draw opacity=1 ]   (162.61,188.37) -- (175.34,161.74) ;
%Straight Lines [id:da24678469450782214] 
\draw [color={rgb, 255:red, 255; green, 0; blue, 4 }  ,draw opacity=1 ]   (179.01,149.65) -- (193.72,119.74) ;
%Straight Lines [id:da5272850339154104] 
\draw [color={rgb, 255:red, 255; green, 0; blue, 4 }  ,draw opacity=1 ]   (197.38,107.66) -- (212.38,75.39) ;
%Straight Lines [id:da2619249126191133] 
\draw    (386.26,77.89) -- (385.6,235.37) ;
%Straight Lines [id:da8449200540079671] 
\draw    (304.9,156.63) -- (464.34,156.63) ;
%Straight Lines [id:da5384424075223206] 
\draw    (306.21,77.23) -- (463.03,234.05) ;
%Shape: Ellipse [id:dp32050838721192254] 
\draw  [fill={rgb, 255:red, 6; green, 0; blue, 0 }  ,fill opacity=1 ] (381.14,156.63) .. controls (381.14,153.98) and (383.29,151.84) .. (385.93,151.84) .. controls (388.58,151.84) and (390.72,153.98) .. (390.72,156.63) .. controls (390.72,159.27) and (388.58,161.42) .. (385.93,161.42) .. controls (383.29,161.42) and (381.14,159.27) .. (381.14,156.63) -- cycle ;
%Shape: Arc [id:dp5174876045526692] 
\draw  [draw opacity=0] (390.87,148.99) .. controls (391.94,149.13) and (392.99,149.46) .. (394,149.99) .. controls (398.62,152.42) and (400.4,158.13) .. (397.98,162.76) .. controls (395.55,167.38) and (389.83,169.16) .. (385.21,166.73) .. controls (384.2,166.2) and (383.33,165.52) .. (382.61,164.73) -- (389.6,158.36) -- cycle ; \draw  [color={rgb, 255:red, 17; green, 10; blue, 255 }  ,draw opacity=1 ] (390.87,148.99) .. controls (391.94,149.13) and (392.99,149.46) .. (394,149.99) .. controls (398.62,152.42) and (400.4,158.13) .. (397.98,162.76) .. controls (395.55,167.38) and (389.83,169.16) .. (385.21,166.73) .. controls (384.2,166.2) and (383.33,165.52) .. (382.61,164.73) ;  
%Shape: Arc [id:dp1608179043401773] 
\draw  [draw opacity=0] (379.81,163.89) .. controls (378.71,163.71) and (377.62,163.33) .. (376.59,162.75) .. controls (371.87,160.09) and (370.19,154.11) .. (372.85,149.38) .. controls (375.51,144.66) and (381.5,142.98) .. (386.22,145.64) .. controls (387.25,146.22) and (388.14,146.96) .. (388.86,147.8) -- (381.41,154.2) -- cycle ; \draw  [color={rgb, 255:red, 255; green, 0; blue, 0 }  ,draw opacity=1 ] (379.81,163.89) .. controls (378.71,163.71) and (377.62,163.33) .. (376.59,162.75) .. controls (371.87,160.09) and (370.19,154.11) .. (372.85,149.38) .. controls (375.51,144.66) and (381.5,142.98) .. (386.22,145.64) .. controls (387.25,146.22) and (388.14,146.96) .. (388.86,147.8) ;  
%Straight Lines [id:da6481286545984176] 
\draw [color={rgb, 255:red, 255; green, 0; blue, 4 }  ,draw opacity=1 ]   (332.56,211.33) -- (379.81,163.89) ;
%Straight Lines [id:da7073051683838827] 
\draw [color={rgb, 255:red, 38; green, 0; blue, 255 }  ,draw opacity=1 ]   (332.56,211.33) -- (382.61,164.73) ;
%Straight Lines [id:da9133478094393831] 
\draw [color={rgb, 255:red, 38; green, 0; blue, 255 }  ,draw opacity=1 ]   (390.87,148.99) -- (440.92,102.38) ;
%Straight Lines [id:da6255403347225865] 
\draw [color={rgb, 255:red, 255; green, 0; blue, 4 }  ,draw opacity=1 ]   (388.86,147.8) -- (440.92,102.38) ;

% Text Node
\draw (118.65,221.73) node [anchor=north west][inner sep=0.75pt]    {$A$};
% Text Node
\draw (233.67,66.87) node [anchor=north west][inner sep=0.75pt]    {$B$};
% Text Node
\draw (141.06,122.8) node [anchor=north west][inner sep=0.75pt]  [font=\footnotesize]  {$\textcolor[rgb]{1,0,0}{\gamma _{AB}^{L}}$};
% Text Node
\draw (199.46,154.3) node [anchor=north west][inner sep=0.75pt]  [font=\footnotesize]  {$\textcolor[rgb]{0,0.14,1}{\gamma _{AB}^{R}}$};
% Text Node
\draw (309.79,92.18) node [anchor=north west][inner sep=0.75pt]  [font=\scriptsize]  {$3$};
% Text Node
\draw (305.19,158.46) node [anchor=north west][inner sep=0.75pt]  [font=\scriptsize]  {$1$};
% Text Node
\draw (372.78,79.72) node [anchor=north west][inner sep=0.75pt]  [font=\scriptsize]  {$2$};
% Text Node
\draw (372.78,218.04) node [anchor=north west][inner sep=0.75pt]  [font=\scriptsize]  {$2$};
% Text Node
\draw (431.84,215.41) node [anchor=north west][inner sep=0.75pt]  [font=\scriptsize]  {$3$};
% Text Node
\draw (456.11,157.8) node [anchor=north west][inner sep=0.75pt]  [font=\scriptsize]  {$1$};
% Text Node
\draw (319.6,179.3) node [anchor=north west][inner sep=0.75pt]  [font=\tiny]  {$\textcolor[rgb]{1,0,0}{\gamma }\textcolor[rgb]{1,0,0}{_{AB}^{L}}$};
% Text Node
\draw (345.19,198.98) node [anchor=north west][inner sep=0.75pt]  [font=\tiny]  {$\textcolor[rgb]{0,0.14,1}{\gamma }\textcolor[rgb]{0,0.14,1}{_{AB}^{R}}$};


\end{tikzpicture}


    \caption{Example of a left and right deformation of a line from point $A$ to point $B$ (left), and the intersections of each around a lattice point on the labeled triangulated torus lattice (right).}
\end{figure}
Notice that if $l_{AB}$ does not intersect any lattice point, then $l_{AB}= \gamma_{AB}^L = \gamma_{AB}^R$. Moreover we have the following;
\begin{theorem}
    Let $A$ and $B$ be two lattice points and let $\gamma_{AB}^L, \gamma_{AB}^R$ be the left and right-deformation of a line $l_{AB}$. Then,
    \begin{equation}
        |\gamma_{AB}^L| = |\gamma_{AB}^R|;
    \end{equation}
    where $|\gamma|$ is the number of perfect matchings of the snake graph $\s_{\gamma}$, known as the \emph{length} of $\gamma$. 
\end{theorem}
By the above theorem, we now define the following concept;
\begin{definition}
    The \emph{Markov distance} $|AB|$ between two lattice points $A,B$ is 
\begin{equation*}
    |AB| = |\gamma_{AB}^L|.
\end{equation*}
\end{definition}
With this definition if we consider a line from the origin $O$ to a point $A = (q,p)$ with slope $p/q$, then $|OA|$ is the number of perfect matchings of the snake graph $\s_{\gamma_{OA}^L}$; denote this by $m_{q,p}$. Observe that if $p$ and $q$ are coprime, then $m_{q,p}$ is a Markov number; more precisely, $m_{q,p} = m_{p/q}$. 

By a very involved argument through Skein relations, the authors in \cite{KLRS} proved that for any two lattice points $A,B$, given any arc $\gamma$ from $A$ to $B$, then $|AB| \leq |\gamma|$. This yields the following corollary, also known as Ptolemy's inequality;
\begin{corollary}\label{Ptolemy}
Given any four points $A,B,C,D$ in the plane such that the lines $l_{AB},l_{BC},l_{CD},l_{DA}$ form a convex quadrilateral with diagonals $l_{AC}$ and $l_{BD}$, then we have 
\begin{equation*}
    |AC||BD| \geq |AB||CD| + |AD||BC|.
\end{equation*}
\end{corollary}
\begin{figure}[H]
    \centering
    

\tikzset{every picture/.style={line width=0.75pt}} %set default line width to 0.75pt        

\begin{tikzpicture}[x=0.75pt,y=0.75pt,yscale=-1,xscale=1]
%uncomment if require: \path (0,398); %set diagram left start at 0, and has height of 398

%Straight Lines [id:da9227894635307007] 
\draw    (71.3,197.9) -- (164.3,59.9) ;
%Straight Lines [id:da7896190622253134] 
\draw    (71.3,197.9) -- (159.3,240.9) ;
%Straight Lines [id:da51296016506204] 
\draw    (297.3,112.9) -- (159.3,240.9) ;
%Straight Lines [id:da6420534118904038] 
\draw    (164.3,59.9) -- (297.3,112.9) ;
%Shape: Circle [id:dp0935769664392595] 
\draw  [fill={rgb, 255:red, 5; green, 0; blue, 0 }  ,fill opacity=1 ] (70.3,196.9) .. controls (70.3,194.94) and (71.89,193.35) .. (73.85,193.35) .. controls (75.81,193.35) and (77.4,194.94) .. (77.4,196.9) .. controls (77.4,198.86) and (75.81,200.45) .. (73.85,200.45) .. controls (71.89,200.45) and (70.3,198.86) .. (70.3,196.9) -- cycle ;
%Shape: Circle [id:dp5515883218329574] 
\draw  [fill={rgb, 255:red, 5; green, 0; blue, 0 }  ,fill opacity=1 ] (161.75,60.9) .. controls (161.75,58.94) and (163.34,57.35) .. (165.3,57.35) .. controls (167.26,57.35) and (168.85,58.94) .. (168.85,60.9) .. controls (168.85,62.86) and (167.26,64.45) .. (165.3,64.45) .. controls (163.34,64.45) and (161.75,62.86) .. (161.75,60.9) -- cycle ;
%Shape: Circle [id:dp4494367756701162] 
\draw  [fill={rgb, 255:red, 5; green, 0; blue, 0 }  ,fill opacity=1 ] (155.75,239.9) .. controls (155.75,237.94) and (157.34,236.35) .. (159.3,236.35) .. controls (161.26,236.35) and (162.85,237.94) .. (162.85,239.9) .. controls (162.85,241.86) and (161.26,243.45) .. (159.3,243.45) .. controls (157.34,243.45) and (155.75,241.86) .. (155.75,239.9) -- cycle ;
%Shape: Circle [id:dp5793176977375761] 
\draw  [fill={rgb, 255:red, 5; green, 0; blue, 0 }  ,fill opacity=1 ] (293.75,113.35) .. controls (293.75,111.39) and (295.34,109.8) .. (297.3,109.8) .. controls (299.26,109.8) and (300.85,111.39) .. (300.85,113.35) .. controls (300.85,115.31) and (299.26,116.9) .. (297.3,116.9) .. controls (295.34,116.9) and (293.75,115.31) .. (293.75,113.35) -- cycle ;
%Straight Lines [id:da2684140690246839] 
\draw [color={rgb, 255:red, 255; green, 0; blue, 0 }  ,draw opacity=1 ]   (77.4,196.9) -- (293.75,113.35) ;
%Straight Lines [id:da8310983432316216] 
\draw [color={rgb, 255:red, 0; green, 28; blue, 255 }  ,draw opacity=1 ]   (159.3,236.35) -- (165.3,64.45) ;

% Text Node
\draw (56,172.4) node [anchor=north west][inner sep=0.75pt]    {$A$};
% Text Node
\draw (142,53.4) node [anchor=north west][inner sep=0.75pt]    {$B$};
% Text Node
\draw (288,119.4) node [anchor=north west][inner sep=0.75pt]    {$C$};
% Text Node
\draw (178,227.4) node [anchor=north west][inner sep=0.75pt]    {$D$};


\end{tikzpicture}

    \caption{Ptolemy's relations.}
\end{figure}
Through the above, we can now state and prove one of Aigner's conjectures in \cite{A};
\begin{theorem}
    For all integers $0 \leq p \leq q$ we have the following:
    \begin{align}
        m_{q,p} &< m_{q,p+1} \\
        m_{q,p} &< m_{q+1,p} \\
        m_{q,p} &< m_{q+1,p-1}
    \end{align}
\end{theorem}
\begin{proof}
    Let $A = (0,0), B = (q,p), C = (q+1,p)$ and $D = (q+1,p-1)$, then by Ptolemy's relations, we have that
    \begin{equation*}
        |AC||BD| \geq |AB||CD|+|AD|+|BC|.
    \end{equation*}
    Since $BD,CD,BC$ are arcs in our triangulation, the number of perfect matchings of their corresponding snake graph is exactly 1; i.e. $|BC|=|CD|=|BD|=1$, and so we get $|AC| \leq |AB|+|AD|$; which yields that $|AC| > |AB|$ and $|AC| > |AD|$; which prove the (4.4) and (4.5); respectively. For 4.6, let $E = (q-p+1,0)$; then $B-E = (p-1,p)$ and $D-E = (p,p-1)$; so $|DE| = |BE|$; and if we consider the quadrilateral with vertices $A,B,D,E$, via Ptolemy's relations we get
    \begin{equation*}
        |AD||BE| \geq |AB||DE| + |BD||AE|;
    \end{equation*}
    which gives that $|AD||BE| > |AB||DE|$. Which proves (4.6).
\end{proof}

\begin{figure}[H]
\centering
\begin{tikzpicture}
\begin{pgfonlayer}{bg}
    \clip (0,0) rectangle (9cm,7cm);
    \coordinate (Origin)   at (0,0);
    \coordinate (XAxisMin) at (0,0);
    \coordinate (XAxisMax) at (9,0);
    \coordinate (YAxisMin) at (0,0);
    \coordinate (YAxisMax) at (0,7);
    \draw[style=help lines] (-14,-14) grid[step=1cm] (14,14);
\end{pgfonlayer}   

    \draw (-0.5,0) node [anchor=north west][inner sep=0.75pt]    {$A$};
    \draw (7.85,7.5) node [anchor=north west][inner sep=0.75pt]    {$B$};
    \draw (9,7.5) node [anchor=north west][inner sep=0.75pt]    {$C$};
    \draw (9.25,6.2) node [anchor=north west][inner sep=0.75pt]    {$D$};
    \draw (4,-0.2) node [anchor=north west][inner sep=0.75pt]    {$E$};
    \draw [very thick, black,-latex] (XAxisMin) -- (XAxisMax);% Draw x axis
     \draw [very thick, black,-latex] (YAxisMin) -- (YAxisMax);% Draw y axis
    \foreach \x in {0,1,...,9}{                           % Two indices running over each
    \foreach \y in {0,1,...,7}{                       % node on the grid we have drawn 
    \node[draw,circle,inner sep=0.6pt,fill] at (\x,\y) {}; % Places a dot at those points
    
    \foreach \a in {0,1,...,8}{
    \foreach \b in {0,1,...,6}{
    \draw[gray, very thin] (\a, \b+1) -- (\a+1,\b);


    \draw[thick] (0,0) -- (8,7) -- (9,7) -- (9,6) -- (0,0);
    \draw[red, thick] (8,7) -- (9,6);
    \draw[blue,thick] (0,0) -- (9,7); 
    \draw[thick] (4,0) -- (9,7); 
    
}
}
}
}
\end{tikzpicture}
\end{figure}
For example, consider the pair $(p,q) = (7,11)$. Observe that we have $m_{11,7} = m_{7/11} = 3276509$, and
\begin{equation*}
    m_{11,8} = m_{8/11} = 7453378; \ m_{12,7} = m_{7/12} = 8399329; \ m_{12,6} = 3729600.
\end{equation*}
Thus, the inequalities from the theorem above are satisfied.
\chapter{Conclusion}
In this paper we looked at the foundational ideas of Cluster algebras, specifically those of surface type; and we discussed the combinatorial aspects such as snake graphs and Skein relations. Via continued fractions we described the concept of palindromification; which we used to construct a framework for understanding how Markov numbers relate to palindromic continued fractions; such as Frobenius' result that states every Markov number is the numerator of the palindromic continued fraction of even length consisting of only $1$s and $2$s (with a few other restrictions). We were then able to prove different results on the structure of each Markov number; such as that every Markov number is the sum of two relatively prime squares. 

Using the once punctured torus, we described a deep connection between solutions of Markov's equation and Cluster algebras, by constructing a map that send a Markov triple to another Markov triple. Via this connection, we stated a Conjecture, purely in Cluster algebraic terms, that is equivalent to Frobenius' conjecture. Finally, thanks to Skein relations, we were able to generalize the idea of a slope on the once punctured torus lattice to slopes with not necessarily relatively prime coordinates, via the concepts of left and right deformations. Using Ptolemy's relations, we were then able to prove one of Aigner's conjectures on the ordering of Markov numbers. In conclusion, we had a close look at how the Markov numbers behave individually as well as in groups; which will hopefully, one day, lead us to a clearer understanding of Frobenius' conjecture and Markov's equation. 
 
\printbibliography
\end{document}